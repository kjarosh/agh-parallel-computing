\documentclass[twocolumn]{article}

\usepackage[hidelinks]{hyperref}
\usepackage[polish]{babel}
\usepackage[backend=biber,style=numeric]{biblatex}
\usepackage{listings}
\usepackage{cleveref}
\usepackage{tikz}
\usepackage{pgfplots}

\lstset{
    captionpos=b,
    frame=single,
    language=Python,
    basicstyle=\footnotesize\ttfamily,
    float,
    floatplacement=H,
}

\title{HPC w Pythonie: Cython i Numba -- porównanie}
\author{Kamil Jarosz \and Wiktor Sus \and Michał Śledź}

\bibliography{bibliography}

\begin{document}
    \maketitle
    \clearpage

    \section{Wstęp}

Python jako język interpretowany na pierwszy rzut oka nie
wydaje się odpowiedni dla dziedzin, w~których dużą rolę
pełni wydajność.
Jedną z~takich dziedzin jest High Performance Computing (HPC).
Okazuje się jednak, że~mimo wad Pythona i~jego teoretycznej wydajności
w porównaniu z językami kompilowanymi,
wkładana jest duża ilość pracy w~próbę jego przyspieszenia
oraz w rozwiązania pozwalające na uzyskanie wydajności porównywalnej
z~innymi, bardziej niskopoziomowymi językami (typu C, C++).

\subsection{Wady i zalety Pythona w HPC}

Python jest językiem interpretowanym, co sprawia że
jego wydajność jest bardzo niska w porównaniu z innymi językami.
W~przypadku pisania kodu, który jest relatywnie często wykonywany,
może to mieć bardzo duży wpływ na czas wykonania obliczeń.
Mimo tego faktu, jest on niesamowicie popularny
nie tylko w zastosowaniach czysto skryptowych ale również w bardziej
wymagających dziedzinach, takich jak data science.
Zajmuje trzecie miejsce wśród najpopularniejszych
języków programowania~\cite{tiobe}.

Wbrew pozorom, posiada on wiele funkcji i aspektów,
które ułatwiają pisanie programów HPC~\cite{nih_python_in_hpc}.
Jedną z większych i bardziej oczywistych zalet jest
prostota i czytelność kodu.
Próg wejścia jest relatywnie niski w porównaniu z innymi
językami:
Python używany jest często w dziedzinach nauki niezwiązanych
z~informatyką przez ludzi, którzy nie specjalizują się
w~programowaniu.
Ma to duży wpływ na programy równoległe, które często
wymagają użycia skomplikowanych konstrukcji językowych
lub bibliotecznych.
W Pythonie istnieje bardzo duży wachlarz możliwości wykorzystania
konstrukcji językowych do poprawy zwięzłości pisanego kodu:
możliwość nadpisania operatorów, użycie operatora slice,
dekoratory funkcji, oraz wiele innych.
Python też nie ogranicza użytkownika i~nie wymaga od niego
pisania kodu w specyficzny sposób.
Zależnie od problemu jaki jest rozwiązywany kod może
być pisany obiektowo lub proceduralnie.
Dla niektórych zagadnień (mając również na uwadze HPC) często kod proceduralny
jest bardziej naturalnym wyborem, jednak nie znaczy to, że
struktury danych lub inne części kodu nie mogą być pisane obiektowo.

Kolejnym argumentem jest popularność i~wsparcie społeczności.
Popularność Pythona jest wysoka i~ciągle rośnie,
a~co za tym idzie powstaje bardzo dużo narzędzi i~rozwiązań,
które są przeznaczone do wielu dziedzin --- w~tym HPC\@.
Sprawia to też, że duży wysiłek jest wkładany
w ulepszanie istniejących rozwiązań.
Biblioteki takie jak NumPy, SciPy lub Matplotlib są
jednymi z podstawowych bibliotek i~traktowane są jako
podstawa wszelakich obliczeń i~wizualizacji.

Sam fakt, że Python jest interpretowany nie musi
też znaczyć, że wszystkie niskopoziomowe operacje
muszą być w nim zaimplementowane.
Rzeczy takie jak mnożenie macierzy, losowanie liczb,
operacje na wielowymiarowych tablicach są implementowane
w~bardziej niskopoziomowych i kompilowanych językach.
Pozwala to na użycie Pythona do wysokopoziomowego opisu
procedur i algorytmów, zostawiając najbardziej podatne
na optymalizację fragmenty programu w językach bardziej
wydajnych, z~którymi Python umożliwia integrację.
Są to między innymi C/C++~\cite{python_c_cpp} lub
Fortran~\cite{python_f2py}.

Niestety jedną z poważniejszych wad jakie posiada Python
to Global Interpreter Lock (GIL).
Jest to mechanizm, który nie pozwala na współbieżne wykonanie
kodu Pythona.
Istnieje on głównie dlatego, że mechanizm zarządzania
pamięcią CPythona nie ma zabezpieczeń w środowisku wielowątkowym.
Przez istnienie GIL wiele funkcji zostało napisanych
z myślą o nim, przez co zrównoleglanie ich
i~używanie ich w programach równoległych
jest niemożliwe~\cite{python_gil}.

\subsection{Użycie Pythona w HPC}

Dowodem przewagi zalet nad wadami jest popularność Pythona
wśród organizacji i środowisk akademickich
skupiających się na HPC\@, na przykład takich jak:
NIH HPC Group~\cite{nih_hpc},
HPC USC~\cite{hpc_usc}, lub
HPC Center RTU~\cite{hpc_rtu}.

    \section{Opis użytych narzędzi}
Istnieje wiele narzędzi pozwalających na wykonywanie obliczeń w bardziej wydajny sposób w Pythonie. Wybraliśmy dwa z nich, Cythona i Numbę. Nasza decyzja motywowana była popularnością pakietów, jak również znajomość języka C używanego w Cythonie i łatwość obsługi obu bibliotek.
        
\subsection{Cython}
Cython jest nadzbiorem Pythona, który dodatkowo umożliwia wykorzystanie funkcji napisanych w języku C jak również jego typów. To pozwala kompilatorowi na tworzenie bardzo wydajnego kodu w języku C ze źródłowego kodu Cythona. Jest on generowany raz a następnie kompilowany przy użyciu wszystkich bardziej popularnych wersji kompilatora C/C++.

Głównym zastosowaniem Cythona jest optymalizacja złożonych obliczeniowo fragmentów kodu poprzez kompilację ich do języka C lub w całości zastąpieniem natywnym kodem. Niestety dodatkowe ograniczenia narzucane przez statyczne typowanie istotnie komplikują interakcję z bibliotekami napisanymi w Pythonie i często trzeba odpowiednio je integrować z kodem Cythona.

Za:
\begin{itemize}
    \item Łatwość użycia
    \item Wysoka wydajność
    \item Możliwość używania samych dekoratorów jak i pełnego wykorzystania języka C
\end{itemize}

Przeciw:
\begin{itemize}
    \item Integracja kodu Cythona z Pythonowym może być istotnie utrudniona, zwłaszcza podczas używania zewnętrznych bibliotek
    \item Wymaga zainstalowanego dodatkowo kompilatora C/C++
\end{itemize}


\subsection{Numba}
Numba - JIT kompilator Pythona. Pozwala na optymalizację i kompilację  kodu Pythona do kodu maszynowego.

Numba daje najlepsze wyniki przy zastosowaniu jej na zwykłych pętlach oraz kodzie napisanym przy użyciu NumPy’a. Nie wspiera Pandasa. Ma problemy ze zwykłymi funkcjami napisanymi w Pythonie, które nie są typowane statycznie (to jest do sprawdzenia jeszcze). Dobrze współpracuje ze Sparkiem, Daskiem i Jupyterem.

Numby używa się przez zastosowanie odpowiednich dekoratorów. Daje to dużą wygodę podczas implementacji. Podstawowe dekoratory:
@jit - może pracować w trybie nopython lub object. Ustawienie parametru nopython na wartość True powoduje skompilowanie funkcji oznaczonej tym dekoratorem do kodu maszynowego - a co za tym idzie do jej wykonania nie będzie używany interpreter Pythona. W przypadku gdy Numba nie może skompilować kodu całej funkcji  przechodzi do trybu object, w którym identyfikuje tylko pętle, które jest w stanie skompilować, i kompiluje je jako osobne funkcje do kodu maszynowego.  Dla uzyskania największej wydajności należy używać trybu nopython.

Dodatkowe parametry @jit:
\begin{description}
    \item [parallel] - ustawione na True powoduje, że Numba próbuje zrównoleglić oznaczoną funkcję. Wraz z tym parametrem można używać funkcji prange zamiast range. Wskazujemy wtedy explicite, że dana pętla ma być zrównoleglona.
    \item [fastmath]  - ustawione na True powoduje, że kosztem zgodności ze standardem IEEE 754 możemy uzyskać dodatkową wydajność. Jeżelii w środowisku, w którym wykonujemy program dostępna jest biblioteka SVLM to ustawienie tego parametru powoduje obniżenie dokładności obliczeń zwiększając ich wydajność.
    \item [@njit] - alias na @jit(nopython=True)
    \item [@cfunc] - powoduje, że oznaczoną funkcję można wywoływać z kodu napisanego w C/C++
\end{description}

Numby można również używać z powodzeniem do programowania GPU, zarówno Nvidia CUDA jak i AMD ROC (to drugie eksperymentalnie), natomiast programowanie GPU nie było tematem naszej pracy.

    \section{Opis benchmarków}

\subsection{Przybliżanie liczby pi}

Pierwszym benchmarkiem jest przybliżanie liczby pi za pomocą
algorytmu Monte Carlo.
Losujemy zadaną liczbę punktów z kwadratu jednostkowego
i obliczamy jaka część należy do ćwiartki koła.
Algorytm ten jest bardzo łatwy w zrównolegleniu
i pozwala na dobrą reprezentację wyników.

Wadą tego algorytmu jest natomiast ekstensywne wykorzystywanie
generatora liczb pseudolosowych.
W naszym przypadku, gdy korzystamy z wielu narzędzi, może
to oznaczać duże wahania w wynikach w zależności od użytego narzędzia.
Jak się później okaże, różne generatory mogą dawać istotnie różne wyniki,
czasem sięgające dwóch rzędów wielkości.
Problem ten ukazał się już po wykonaniu eksperymentów,
podczas interpretacji wyników.

\subsection{Mnożenie wektorów}

Kolejnym benchmarkiem jest mnożenie dwóch wektorów o rozmiarach
$n$ tak, aby otrzymać macierz rozmiaru $n\times n$.
Wybraliśmy taki test głównie po to, aby porównać wydajność
z biblioteką NumPy bez niepotrzebnej implementacji skomplikowanych
algorytmów.

    \section{Cython (Kamil Jarosz)}

%\subsection{Przybliżanie liczby $\pi$}

\begin{figure*}
    \centering
    \begin{tikzpicture}[trim axis left]
    \begin{axis}[
        xlabel=Rozmiar problemu,
        ylabel=Czas wykonania,
        legend pos=outer north east]
        \addplot table[mark=x, x index=1, y index=0] {results/sizes-seq_cython.dat};
        \addplot table[mark=o, x index=1, y index=0] {results/sizes-seq_native_cython.dat};
        \addplot table[mark=*, x index=1, y index=0] {results/sizes-mp_cython.dat};
        \addplot table[mark=., x index=1, y index=0] {results/sizes-mp_native_cython.dat};

        \legend{Cython, Cython Native, MP Cython, MP Cython Native}
    \end{axis}
\end{tikzpicture}

    \caption{Czas wykonania algorytmu w Cythonie w zależności od rozmiaru problemu}
    \label{fig:cython-sizes}
\end{figure*}

Dla Cythona wybraliśmy głównie dwa warianty:
\begin{itemize}
    \item kod napisany w Pythonie kompilowany do C automatycznie, oraz
    \item kod napisany w C, zintegrowany z Pythonem.
\end{itemize}
Pierwszy wariant oznaczany jest jako ``Cython'', drugi jako ``Cython Native''.
Dodatkowo zbadaliśmy wpływ użycia biblioteki \textit{multiprocessing}
na oba warianty.

Na rysunku~\ref{fig:cython-sizes} przedstawiony jest wykres czasu wykonania
algorytmu, w zależności od rozmiaru problemu.
Warianty z użyciem \textit{multiprocessing} były uruchamiane na
8 wątkach.
Niestety wariant natywny napisany w C jest średnio 20 razy szybszy,
mimo faktu, że oba warianty są kompilowane.
Różnica ta jest spowodowana głównie przez dwie rzeczy:
\begin{itemize}
    \item kod kompilowany przez Cythona ma dosyć istotny narzut związany z
    używaniem Pythonowych obiektów i funkcji,
    \item duży wpływ na czas wykonania programu w algorytmach Monte Carlo
    ma użyty generator liczb pseudolosowych, okazuje się
    bowiem że generator z Pythona jest nieco wolniejszy od
    generatora z biblioteki standardowej C\@.
\end{itemize}
Zauważyć można, że mimo uruchomienia programu na 8 wątkach,
natywny kod sekwencyjny nadal jest szybszy od kodu skompilowanego przez
Cythona.

Wyżej omówione wyniki ukazują jak ważny jest dobór odpowiednich narzędzi
--- zarówno sposobu napisania kodu jak i rzeczy takich jak generator
liczb pseudolosowych.

\begin{figure*}
    \centering
    \begin{minipage}[b]{.45\textwidth}
        \centering
        \begin{tikzpicture}[trim axis left]
    \begin{axis}[
        xlabel=Liczba wątków,
        ylabel=Czas wykonania (s),
        legend pos=north east]
        \addplot table[color=red, mark=x, x index=1, y index=0] {results/threads-mp_cython.dat};
        \addplot table[color=blue, mark=o, x index=1, y index=0] {results/threads-mp_native_cython.dat};

        \legend{Cython, Cython Native}
    \end{axis}
\end{tikzpicture}

        \caption{Czas wykonania algorytmu w Cythonie w zależności od liczby wątków}
        \label{fig:cython-threads}
    \end{minipage}
    \hfill
    \begin{minipage}[b]{.45\textwidth}
        \centering
        \begin{tikzpicture}[trim axis left]
    \begin{axis}[
        xlabel=Liczba wątków,
        ylabel=Przyspieszenie,
        legend pos=north west]
        \addplot table[color=red, mark=x, x index=1, y index=2] {results/threads-mp_cython.dat};
        \addplot table[color=blue, mark=o, x index=1, y index=2] {results/threads-mp_native_cython.dat};

        \addplot[domain=0:9, samples=2, color=gray, style=dashed]{x};

        \legend{Cython, Cython Native}
    \end{axis}
\end{tikzpicture}

        \caption{Przyspieszenie liczenia liczby $\pi$ w Cythonie}
        \label{fig:cython-speedup}
    \end{minipage}
\end{figure*}

Na rysunku~\ref{fig:cython-threads} przedstawiony jest czas wykonania
w zależności od liczby wątków.
Testy zostały wykonane dla wystarczająco dużego rozmiaru problemu,
który był równy \texttt{250\_000\_000} punktów.
Natywny Cython jest szybszy średnio 20 krotnie.

Na rysunku~\ref{fig:cython-speedup} widoczne jest również przyspieszenie
obu wariantów kodu.
W zależności od liczby wątków (1--16) przedstawiamy współczynnik
przyspieszenia programu.
Dla obu przypadków widać, że przyspieszenie rośnie niemalże liniowo.
Dla większej liczby wątków niż 8, widać nieliniowe zachowanie się czasu
wykonania programu.

Na listingu~\ref{lst:cython-seq} widoczny jest użyty kod w wariancie
sekwencyjnym.
Jest tam użyta konstrukcja \texttt{cdef}, która pozwala
oznaczyć funkcję aby została skompilowana przez Cythona.
Na listingu~\ref{lst:cython-native-seq} jest widoczny kod
natywny.
Zdefiniowany jest tam mapping między funkcjami w C a w Pythonie.
Listing zawiera również implementację algorytmu w kodzie C\@.

% @formatter:off
\noindent\begin{minipage}{\columnwidth}
\begin{lstlisting}[
    caption=Kod sekwencyjny w Cythonie,
    label={lst:cython-seq}]
import random

cdef _run(points):
    circle_points = 0
    for _ in range(points):
        x = random.uniform(0, 1)
        y = random.uniform(0, 1)
        if (x - 1) ** 2 + (y - 1) ** 2 < 1:
            circle_points += 1
    return 4 * circle_points / points

def run(points):
    return _run(points)
\end{lstlisting}
\end{minipage}
% @formatter:on

% @formatter:off
\noindent\begin{minipage}{\columnwidth}
\begin{lstlisting}[
    caption=Kod sekwencyjny w natywnym Cythonie,
    label={lst:cython-native-seq}]
cdef extern from *:
    """
    static double _run(int points) {
        int circle_points = 0;
        for (int i = 0; i < points; ++i) {
            double x = (double) rand()
                    / RAND_MAX;
            double y = (double) rand()
                    / RAND_MAX;
            if ((x - 1) * (x - 1) +
                (y - 1) * (y - 1) < 1) {
                ++circle_points;
            }
        }
        return 4.d * circle_points / points;
    }
    """
    double _run(int points)

def run(points):
    return _run(points)
\end{lstlisting}
\end{minipage}
% @formatter:on

%\subsection{Mnożenie wektorów}





    \begin{figure*}
        \centering
        \begin{minipage}[b]{.45\textwidth}
            \centering
            \begin{tikzpicture}[trim axis left]
    \begin{axis}[
        xlabel=Liczba wątków,
        ylabel=Przyspieszenie,
        legend pos=north west]
        \addplot table[x index=1, y index=2] {results/threads-parallel_numba.dat};
        \addplot table[x index=1, y index=2] {results/threads-parallel_fastmath_numba.dat};
        \addplot table[x index=1, y index=2] {results/threads-mp_numba.dat};
        \addplot table[x index=1, y index=2] {results/threads-mp_fastmath_numba.dat};

        \addplot[domain=0:9, samples=2, color=gray, style=dashed]{x};

        \legend{Numba, Numba Fastmath, MP Numba, MP Numba Fastmath}
    \end{axis}
\end{tikzpicture}

            \caption{Przyspieszenie liczenia liczby $\pi$ w Numbie}
            \label{fig:numba-speedup}
        \end{minipage}
        \hfill
        \begin{minipage}[b]{.45\textwidth}
            \centering
            \begin{tikzpicture}[trim axis left]
    \begin{axis}[
        xlabel=Liczba wątków,
        ylabel=Czas wykonania,
        legend pos=north east]
        \addplot table[color=red, mark=x, x index=1, y index=0] {results/threads-parallel_numba.dat};
        \addplot table[color=blue, mark=o, x index=1, y index=0] {results/threads-parallel_fastmath_numba.dat};

        \legend{Numba, Numba Fastmath}
    \end{axis}
\end{tikzpicture}

            \caption{Czas wykonania algorytmu w Numbie w zależności od liczby wątków}
            \label{fig:numba-threads}
        \end{minipage}
    \end{figure*}

    \begin{figure*}
        \centering
        \begin{tikzpicture}[trim axis left]
    \begin{axis}[
        xlabel=Rozmiar problemu,
        ylabel=Czas wykonania,
        legend pos=outer north east]
        \addplot table[mark=x, x index=1, y index=0] {results/sizes-seq_numba.dat};
        \addplot table[mark=o, x index=1, y index=0] {results/sizes-seq_fastmath_numba.dat};
        \addplot table[mark=*, x index=1, y index=0] {results/sizes-parallel_numba.dat};
        \addplot table[mark=., x index=1, y index=0] {results/sizes-parallel_fastmath_numba.dat};

        \legend{Numba, Numba Fastmath, Parallel Numba, Parallel Numba Fastmath}
    \end{axis}
\end{tikzpicture}

        \caption{Czas wykonania algorytmu w Numbie w zależności od rozmiaru problemu}
        \label{fig:numba-sizes}
    \end{figure*}


    \begin{figure*}
        \centering
        \begin{minipage}[b]{.45\textwidth}
            \centering
            \begin{tikzpicture}[trim axis left]
    \begin{axis}[
        xlabel=Liczba wątków,
        ylabel=Przyspieszenie,
        legend pos=south east]
        \addplot table[color=red, mark=x, x index=1, y index=2] {results/threads-mp_python.dat};

        \addplot[domain=0:9, samples=2, color=gray, style=dashed]{x};

        \legend{MP Python}
    \end{axis}
\end{tikzpicture}

            \caption{Przyspieszenie liczenia liczby $\pi$ w czystym Pythonie}
            \label{fig:python-speedup}
        \end{minipage}
        \hfill
        \begin{minipage}[b]{.45\textwidth}
            \centering
            \begin{tikzpicture}[trim axis left]
    \begin{axis}[
        xlabel=Liczba wątków,
        ylabel=Czas wykonania (s),
        legend pos=north east]
        \addplot table[color=red, mark=x, x index=1, y index=0] {results/threads-mp_python.dat};

        \legend{MP Python}
    \end{axis}
\end{tikzpicture}

            \caption{Czas wykonania algorytmu w czystym Pythonie w zależności od liczby wątków}
            \label{fig:python-threads}
        \end{minipage}
    \end{figure*}

    \begin{figure*}
        \centering
        \begin{tikzpicture}[trim axis left]
    \begin{axis}[
        xlabel=Rozmiar problemu,
        ylabel=Czas wykonania (s),
        legend pos=north east]
        \addplot table[mark=x, x index=1, y index=0] {results/sizes-seq_python.dat};
        \addplot table[mark=*, x index=1, y index=0] {results/sizes-mp_python.dat};

        \legend{Python, MP Python}
    \end{axis}
\end{tikzpicture}

        \caption{Czas wykonania algorytmu w czystym Pythonie w zależności od rozmiaru problemu}
        \label{fig:python-sizes}
    \end{figure*}


    \begin{figure*}
        \centering
        \begin{minipage}[b]{.45\textwidth}
            \centering
            \begin{tikzpicture}[trim axis left]
    \begin{axis}[
        xlabel=Liczba wątków,
        ylabel=Przyspieszenie,
        legend pos=south east]
        \addplot table[color=red, mark=x, x index=1, y index=2] {results/threads-mp_cython.dat};
        \addplot table[color=blue, mark=o, x index=1, y index=2] {results/threads-mp_native_cython.dat};
        \addplot table[color=red, mark=x, x index=1, y index=2] {results/threads-parallel_numba.dat};
        \addplot table[color=blue, mark=o, x index=1, y index=2] {results/threads-parallel_fastmath_numba.dat};
        \addplot table[color=red, mark=x, x index=1, y index=2] {results/threads-mp_python.dat};
        \legend{Cython, Cython Native, Numba, Numba Fastmath, MP Python}

        \addplot[domain=0:9, samples=2, color=gray, style=dashed]{x};
    \end{axis}
\end{tikzpicture}

            \caption{Przyspieszenie liczenia liczby $\pi$}
            \label{fig:all-speedup}
        \end{minipage}
        \hfill
        \begin{minipage}[b]{.45\textwidth}
            \centering
            \begin{tikzpicture}[trim axis left]
    \begin{axis}[
        xlabel=Liczba wątków,
        ylabel=Czas wykonania,
        legend pos=north east]
        \addplot table[color=red, mark=x, x index=1, y index=0] {results/threads-mp_cython.dat};
        \addplot table[color=blue, mark=o, x index=1, y index=0] {results/threads-mp_native_cython.dat};
        \addplot table[color=red, mark=x, x index=1, y index=0] {results/threads-parallel_numba.dat};
        \addplot table[color=blue, mark=o, x index=1, y index=0] {results/threads-parallel_fastmath_numba.dat};
        \addplot table[color=red, mark=x, x index=1, y index=0] {results/threads-mp_python.dat};
        \legend{Cython, Cython Native, Numba, Numba Fastmath, MP Python}
    \end{axis}
\end{tikzpicture}

            \caption{Czas wykonania algorytmu w zależności od liczby wątków}
            \label{fig:all-threads}
        \end{minipage}
    \end{figure*}

    \begin{figure*}
        \centering
        \begin{tikzpicture}[trim axis left]
    \begin{loglogaxis}[
        xlabel=Rozmiar problemu,
        ylabel=Czas wykonania,
        legend pos=outer north east]
        \addplot table[x index=1, y index=0] {results/sizes-seq_cython.dat};
        \addplot table[x index=1, y index=0] {results/sizes-seq_native_cython.dat};
        \addplot table[x index=1, y index=0] {results/sizes-seq_numba.dat};
        \addplot table[x index=1, y index=0] {results/sizes-seq_python.dat};
        \legend{Cython, Cython Native, Numba, Python, MP Python}
    \end{loglogaxis}
\end{tikzpicture}

        \caption{Czas wykonania algorytmu w zależności od rozmiaru problemu}
        \label{fig:all-sizes-seq}
    \end{figure*}

    \begin{figure*}
        \centering
        \begin{tikzpicture}[trim axis left]
    \begin{loglogaxis}[
        xlabel=Rozmiar problemu,
        ylabel=Czas wykonania,
        legend pos=outer north east]
        \addplot table[mark=*, x index=1, y index=0] {results/sizes-mp_cython.dat};
        \addplot table[mark=., x index=1, y index=0] {results/sizes-mp_native_cython.dat};
        \addplot table[mark=*, x index=1, y index=0] {results/sizes-parallel_numba.dat};
        \addplot table[mark=., x index=1, y index=0] {results/sizes-parallel_fastmath_numba.dat};
        \addplot table[mark=*, x index=1, y index=0] {results/sizes-mp_python.dat};

        \legend{MP Cython, MP Cython Native, Parallel Numba, Parallel Numba Fastmath, MP Python}
    \end{loglogaxis}
\end{tikzpicture}

        \caption{Czas wykonania algorytmu w zależności od rozmiaru problemu}
        \label{fig:all-sizes-parallel}
    \end{figure*}


    \printbibliography
\end{document}
