\section{Cython (Kamil Jarosz)}

\subsection{Przybliżanie liczby $\pi$}

\begin{figure*}
    \centering
    \begin{tikzpicture}[trim axis left]
    \begin{axis}[
        xlabel=Rozmiar problemu,
        ylabel=Czas wykonania,
        legend pos=outer north east]
        \addplot table[mark=x, x index=1, y index=0] {results/sizes-seq_cython.dat};
        \addplot table[mark=o, x index=1, y index=0] {results/sizes-seq_native_cython.dat};
        \addplot table[mark=*, x index=1, y index=0] {results/sizes-mp_cython.dat};
        \addplot table[mark=., x index=1, y index=0] {results/sizes-mp_native_cython.dat};

        \legend{Cython, Cython Native, MP Cython, MP Cython Native}
    \end{axis}
\end{tikzpicture}

    \caption{Czas wykonania algorytmu w Cythonie w zależności od rozmiaru problemu}
    \label{fig:cython-sizes}
\end{figure*}

Dla Cythona wybraliśmy głównie dwa warianty:
\begin{itemize}
    \item kod napisany w Pythonie kompilowany do C automatycznie, oraz
    \item kod napisany w C, zintegrowany z Pythonem.
\end{itemize}
Pierwszy wariant oznaczany jest jako ``Cython'', drugi jako ``Cython Native''.
Dodatkowo zbadaliśmy wpływ użycia biblioteki \textit{multiprocessing}
na oba warianty.

Na rysunku~\ref{fig:cython-sizes} przedstawiony jest wykres czasu wykonania
algorytmu, w zależności od rozmiaru problemu.
Warianty z użyciem \textit{multiprocessing} były uruchamiane na
8 wątkach.
Niestety wariant natywny napisany w C jest średnio 20 razy szybszy,
mimo faktu, że oba warianty są kompilowane.
Różnica ta jest spowodowana głównie przez dwie rzeczy:
\begin{itemize}
    \item kod kompilowany przez Cythona ma dosyć istotny narzut związany z
    używaniem Pythonowych obiektów i funkcji,
    \item duży wpływ na czas wykonania programu w algorytmach Monte Carlo
    ma użyty generator liczb pseudolosowych, okazuje się
    bowiem że generator z Pythona jest nieco wolniejszy od
    generatora z biblioteki standardowej C\@.
\end{itemize}
Zauważyć można, że mimo uruchomienia programu na 8 wątkach,
natywny kod sekwencyjny nadal jest szybszy od kodu skompilowanego przez
Cythona.

Wyżej omówione wyniki ukazują jak ważny jest dobór odpowiednich narzędzi
--- zarówno sposobu napisania kodu jak i rzeczy takich jak generator
liczb pseudolosowych.

\begin{figure*}
    \centering
    \begin{minipage}[b]{.45\textwidth}
        \centering
        \begin{tikzpicture}[trim axis left]
    \begin{axis}[
        xlabel=Liczba wątków,
        ylabel=Czas wykonania (s),
        legend pos=north east]
        \addplot table[color=red, mark=x, x index=1, y index=0] {results/threads-mp_cython.dat};
        \addplot table[color=blue, mark=o, x index=1, y index=0] {results/threads-mp_native_cython.dat};

        \legend{Cython, Cython Native}
    \end{axis}
\end{tikzpicture}

        \caption{Czas wykonania algorytmu w Cythonie w zależności od liczby wątków}
        \label{fig:cython-threads}
    \end{minipage}
    \hfill
    \begin{minipage}[b]{.45\textwidth}
        \centering
        \begin{tikzpicture}[trim axis left]
    \begin{axis}[
        xlabel=Liczba wątków,
        ylabel=Przyspieszenie,
        legend pos=north west]
        \addplot table[color=red, mark=x, x index=1, y index=2] {results/threads-mp_cython.dat};
        \addplot table[color=blue, mark=o, x index=1, y index=2] {results/threads-mp_native_cython.dat};

        \addplot[domain=0:9, samples=2, color=gray, style=dashed]{x};

        \legend{Cython, Cython Native}
    \end{axis}
\end{tikzpicture}

        \caption{Przyspieszenie liczenia liczby $\pi$ w Cythonie}
        \label{fig:cython-speedup}
    \end{minipage}
\end{figure*}

Na rysunku~\ref{fig:cython-threads} przedstawiony jest czas wykonania
w zależności od liczby wątków.
Testy zostały wykonane dla wystarczająco dużego rozmiaru problemu,
który był równy \texttt{250\_000\_000} punktów.
Natywny Cython jest szybszy średnio 20 krotnie.

Na rysunku~\ref{fig:cython-speedup} widoczne jest również przyspieszenie
obu wariantów kodu.
W zależności od liczby wątków (1--16) przedstawiamy współczynnik
przyspieszenia programu.
Dla obu przypadków widać, że przyspieszenie rośnie niemalże liniowo.
Dla większej liczby wątków niż 8, widać nieliniowe zachowanie się czasu
wykonania programu.

Na listingu~\ref{lst:cython-seq} widoczny jest użyty kod w wariancie
sekwencyjnym.
Jest tam użyta konstrukcja \texttt{cdef}, która pozwala
oznaczyć funkcję aby została skompilowana przez Cythona.
Na listingu~\ref{lst:cython-native-seq} jest widoczny kod
natywny.
Zdefiniowany jest tam mapping między funkcjami w C a w Pythonie.
Listing zawiera również implementację algorytmu w kodzie C\@.

% @formatter:off
\noindent\begin{minipage}{\columnwidth}
\begin{lstlisting}[
    caption=Kod sekwencyjny w Cythonie,
    label={lst:cython-seq}]
import random

cdef _run(points):
    circle_points = 0
    for _ in range(points):
        x = random.uniform(0, 1)
        y = random.uniform(0, 1)
        if (x - 1) ** 2 + (y - 1) ** 2 < 1:
            circle_points += 1
    return 4 * circle_points / points

def run(points):
    return _run(points)
\end{lstlisting}
\end{minipage}
% @formatter:on

% @formatter:off
\noindent\begin{minipage}{\columnwidth}
\begin{lstlisting}[
    caption=Kod sekwencyjny w natywnym Cythonie,
    label={lst:cython-native-seq}]
cdef extern from *:
    """
    static double _run(int points) {
        int circle_points = 0;
        for (int i = 0; i < points; ++i) {
            double x = (double) rand()
                    / RAND_MAX;
            double y = (double) rand()
                    / RAND_MAX;
            if ((x - 1) * (x - 1) +
                (y - 1) * (y - 1) < 1) {
                ++circle_points;
            }
        }
        return 4.d * circle_points / points;
    }
    """
    double _run(int points)

def run(points):
    return _run(points)
\end{lstlisting}
\end{minipage}
% @formatter:on

%\subsection{Mnożenie wektorów}
