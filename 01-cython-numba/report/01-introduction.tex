\section{Wstęp}

Python jako język interpretowany na pierwszy rzut oka nie
wydaje się odpowiedni dla dziedzin, w~których dużą rolę
pełni wydajność.
Jedną z~takich dziedzin jest High Performance Computing (HPC).
Okazuje się jednak, że~mimo wad Pythona i~jego teoretycznej wydajności
w porównaniu z językami kompilowanymi,
wkładana jest duża ilość pracy w~próbę jego przyspieszenia
oraz w rozwiązania pozwalające na uzyskanie wydajności porównywalnej
z~innymi, bardziej niskopoziomowymi językami (typu C, C++).

\subsection{Wady i zalety Pythona w HPC}

Python jest językiem interpretowanym, co sprawia że
jego wydajność jest bardzo niska w porównaniu z innymi językami.
W~przypadku pisania kodu, który jest relatywnie często wykonywany,
może to mieć bardzo duży wpływ na czas wykonania obliczeń.
Mimo tego faktu, jest on niesamowicie popularny
nie tylko w zastosowaniach czysto skryptowych ale również w bardziej
wymagających dziedzinach, takich jak data science.
Zajmuje trzecie miejsce wśród najpopularniejszych
języków programowania~\cite{tiobe}.

Wbrew pozorom, posiada on wiele funkcji i aspektów,
które ułatwiają pisanie programów HPC~\cite{nih_python_in_hpc}.
Jedną z większych i bardziej oczywistych zalet jest
prostota i czytelność kodu.
Próg wejścia jest relatywnie niski w porównaniu z innymi
językami:
Python używany jest często w dziedzinach nauki niezwiązanych
z~informatyką przez ludzi, którzy nie specjalizują się
w~programowaniu.
Ma to duży wpływ na programy równoległe, które często
wymagają użycia skomplikowanych konstrukcji językowych
lub bibliotecznych.
W Pythonie istnieje bardzo duży wachlarz możliwości wykorzystania
konstrukcji językowych do poprawy zwięzłości pisanego kodu:
możliwość nadpisania operatorów, użycie operatora slice,
dekoratory funkcji, oraz wiele innych.
Python też nie ogranicza użytkownika i~nie wymaga od niego
pisania kodu w specyficzny sposób.
Zależnie od problemu jaki jest rozwiązywany kod może
być pisany obiektowo lub proceduralnie.
Dla niektórych zagadnień (mając również na uwadze HPC) często kod proceduralny
jest bardziej naturalnym wyborem, jednak nie znaczy to, że
struktury danych lub inne części kodu nie mogą być pisane obiektowo.

Kolejnym argumentem jest popularność i~wsparcie społeczności.
Popularność Pythona jest wysoka i~ciągle rośnie,
a~co za tym idzie powstaje bardzo dużo narzędzi i~rozwiązań,
które są przeznaczone do wielu dziedzin --- w~tym HPC\@.
Sprawia to też, że duży wysiłek jest wkładany
w ulepszanie istniejących rozwiązań.
Biblioteki takie jak NumPy, SciPy lub Matplotlib są
jednymi z podstawowych bibliotek i~traktowane są jako
podstawa wszelakich obliczeń i~wizualizacji.

Sam fakt, że Python jest interpretowany nie musi
też znaczyć, że wszystkie niskopoziomowe operacje
muszą być w nim zaimplementowane.
Rzeczy takie jak mnożenie macierzy, losowanie liczb,
operacje na wielowymiarowych tablicach są implementowane
w~bardziej niskopoziomowych i kompilowanych językach.
Pozwala to na użycie Pythona do wysokopoziomowego opisu
procedur i algorytmów, zostawiając najbardziej podatne
na optymalizację fragmenty programu w językach bardziej
wydajnych, z~którymi Python umożliwia integrację.
Są to między innymi C/C++~\cite{python_c_cpp} lub
Fortran~\cite{python_f2py}.

Niestety jedną z poważniejszych wad jakie posiada Python
to Global Interpreter Lock (GIL).
Jest to mechanizm, który nie pozwala na współbieżne wykonanie
kodu Pythona.
Istnieje on głównie dlatego, że mechanizm zarządzania
pamięcią CPythona nie ma zabezpieczeń w środowisku wielowątkowym.
Przez istnienie GIL wiele funkcji zostało napisanych
z myślą o nim, przez co zrównoleglanie ich
i~używanie ich w programach równoległych
jest niemożliwe~\cite{python_gil}.

\subsection{Użycie Pythona w HPC}

Dowodem przewagi zalet nad wadami jest popularność Pythona
wśród organizacji i środowisk akademickich
skupiających się na HPC\@, na przykład takich jak:
NIH HPC Group~\cite{nih_hpc},
HPC USC~\cite{hpc_usc}, lub
HPC Center RTU~\cite{hpc_rtu}.
