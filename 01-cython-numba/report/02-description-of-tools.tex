\section{Opis użytych narzędzi}

Istnieje wiele narzędzi pozwalających na wykonywanie obliczeń w bardziej
wydajny sposób w Pythonie~\cite{python_marowka}.
Wybraliśmy dwa z nich: Cythona i Numbę.
Nasza decyzja motywowana była popularnością pakietów,
jak również znajomością języka C używanego w Cythonie
i łatwością obsługi obu bibliotek.

% @formatter:off
\begin{lstlisting}[
    caption=Przykład funkcji w Cythonie,
    label={lst:cython_example}]
import random

cdef _run(points):
    circle_points = 0
    for _ in range(points):
        x = random.uniform(0, 1)
        y = random.uniform(0, 1)
        if (x - 1) ** 2 + (y - 1) ** 2 < 1:
            circle_points += 1
    return 4 * circle_points / points

def run(points):
    return _run(points)
\end{lstlisting}
% @formatter:on

\subsection{Cython}

Cython jest nadzbiorem Pythona, który dodatkowo umożliwia integrację
z językiem C\@.
Pozwala na użycie funkcji i typów z C jak i udostępnia
możliwość operowania na obiektach z Pythona, w przypadku
pisania kodu w C\@.
Kompilator Cythona zamienia jego kod do postaci języka C,
który później jest kompilowany i optymalizowany
przez wybrany kompilator (na przykład gcc).
Jest to robione raz --- przy pierwszym uruchomieniu,
ewentualnie ręcznie na żądanie.

Głównym zastosowaniem Cythona jest ogólne przyspieszenie programów
poprzez skompilowanie ich do postaci kodu maszynowego.
Następnie kod maszynowy jest wykonywany.
Programista ma możliwość zastąpienia fragmentów kodu, które
są najczęściej wykonywane na postaci bardziej optymalne,
napisane w C\@.
Możliwa jest też łatwa integracja z~istniejącymi skompilowanymi
bibliotekami.

Możliwe jest również użycie kodu napisanego w Pythonie z języka C\@.
Pozwala to na obustronną integrację tych języków.
Niestety dodatkowe ograniczenia narzucane przez statyczne typowanie
istotnie komplikują interakcję z kodem napisanym w Pythonie
i~często trzeba odpowiednio je integrować.

Aby wykorzystać Cythona, należy w kodzie użyć konstrukcji \texttt{cdef}
(\cref{lst:cython_example}).
Oznacza ona daną funkcję, aby skompilowana została do postaci kodu C\@.

Do niewątpliwych zalet Cythona należą:
\begin{itemize}
    \item Łatwość użycia.
    \item Duża kontrola nad postacią kodu wyjściowego.
    \item Swoboda w użyciu: automatyczne tłumaczenie Pythona na C,
    używanie bezpośrednio funkcji z C w Pythonie, lub uruchamianie
    kodu natywnego.
    \item Możliwość automatycznej kompilacji kodu przy imporcie.
\end{itemize}

\subsection{Numba}

Numba jest kompilatorem JIT Pythona~\cite{numba}.
Pozwala na optymalizację i kompilację kodu Pythona do kodu maszynowego
w trakcie jego wykonania z użyciem LLVM~\cite{llvm}.
Jest zintegrowana z NumPy'em oraz z niektórymi funkcjami z biblioteki
standardowej, odpowiednio je optymalizując.
Dobrze współpracuje ze Sparkiem, Daskiem i Jupyterem.

Numba jest bardzo łatwa w użyciu --- wystarczy użyć desygnowanych
adnotacji.
Funkcję nimi oznaczone będą uruchamiane za pomocą Numby.
\begin{description}
    \item[@jit]
    Podstawowa adnotacja pozwalająca na użycie Numby.
    Jej parametry opisane są niżej.
    \item[@njit]
    Równoważne z \texttt{@jit(nopython=True)}.
    \item[@cfunc]
    Powoduje, że oznaczoną funkcję można wywoływać z kodu napisanego w C/C++.
\end{description}
Dodatkowo adnotacja \texttt{@jit} posiada parametry:
\begin{description}
    \item [nopython]
    Numba może pracować w trybie \textit{nopython} lub \textit{object}.
    Ustawienie parametru \textit{nopython} na wartość \texttt{True}
    powoduje skompilowanie
    funkcji oznaczonej tym dekoratorem do kodu maszynowego, a co za tym idzie
    do jej wykonania nie będzie używany interpreter Pythona.
    W przypadku gdy Numba nie może skompilować kodu całej funkcji
    przechodzi do trybu \textit{object}, w którym identyfikuje tylko pętle,
    które jest w stanie skompilować, i kompiluje je jako osobne funkcje
    do kodu maszynowego.
    Dla uzyskania największej wydajności należy używać trybu \textit{nopython}.
    \item [parallel]
    Ustawione na \texttt{True} powoduje, że Numba próbuje
    zrównoleglić oznaczoną funkcję.
    Wraz z tym parametrem można używać funkcji \texttt{prange}
    zamiast \texttt{range}.
    Wskazujemy wtedy, że dana pętla ma być zrównoleglona.
    \item [fastmath]
    Ustawione na \texttt{True} powoduje, że kosztem zgodności
    ze standardem IEEE 754 możemy uzyskać dodatkową wydajność.
    Jeżeli w środowisku, w którym wykonujemy program dostępna jest
    biblioteka SVLM to ustawienie tego parametru powoduje obniżenie
    dokładności obliczeń zwiększając ich wydajność~\cite{numba_fastmnath}.
\end{description}

Numby można również używać z powodzeniem do programowania GPU,
zarówno Nvidia CUDA jak i~AMD~ROC (eksperymentalnie),
natomiast samo programowanie GPU nie było tematem naszej pracy.
