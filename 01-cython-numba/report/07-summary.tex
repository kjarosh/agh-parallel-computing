\begin{figure*}
    \centering
    \begin{tikzpicture}[trim axis left]
    \begin{loglogaxis}[
        xlabel=Rozmiar problemu,
        ylabel=Czas wykonania,
        legend pos=outer north east]
        \addplot table[x index=1, y index=0] {results/sizes-seq_cython.dat};
        \addplot table[x index=1, y index=0] {results/sizes-seq_native_cython.dat};
        \addplot table[x index=1, y index=0] {results/sizes-seq_numba.dat};
        \addplot table[x index=1, y index=0] {results/sizes-seq_python.dat};
        \legend{Cython, Cython Native, Numba, Python, MP Python}
    \end{loglogaxis}
\end{tikzpicture}

    \caption{Czas wykonania algorytmu w zależności od rozmiaru problemu}
    \label{fig:all-sizes-seq}
\end{figure*}


\begin{figure*}
    \centering
    \begin{tikzpicture}[trim axis left]
    \begin{loglogaxis}[
        xlabel=Rozmiar problemu,
        ylabel=Czas wykonania,
        legend pos=outer north east]
        \addplot table[mark=*, x index=1, y index=0] {results/sizes-mp_cython.dat};
        \addplot table[mark=., x index=1, y index=0] {results/sizes-mp_native_cython.dat};
        \addplot table[mark=*, x index=1, y index=0] {results/sizes-parallel_numba.dat};
        \addplot table[mark=., x index=1, y index=0] {results/sizes-parallel_fastmath_numba.dat};
        \addplot table[mark=*, x index=1, y index=0] {results/sizes-mp_python.dat};

        \legend{MP Cython, MP Cython Native, Parallel Numba, Parallel Numba Fastmath, MP Python}
    \end{loglogaxis}
\end{tikzpicture}

    \caption{Czas wykonania algorytmu w zależności od rozmiaru problemu}
    \label{fig:all-sizes-parallel}
\end{figure*}

\begin{figure*}
    \centering
    \begin{minipage}[b]{.45\textwidth}
        \centering
        \begin{tikzpicture}[trim axis left]
    \begin{axis}[
        xlabel=Liczba wątków,
        ylabel=Czas wykonania,
        legend pos=north east]
        \addplot table[color=red, mark=x, x index=1, y index=0] {results/threads-mp_cython.dat};
        \addplot table[color=blue, mark=o, x index=1, y index=0] {results/threads-mp_native_cython.dat};
        \addplot table[color=red, mark=x, x index=1, y index=0] {results/threads-parallel_numba.dat};
        \addplot table[color=blue, mark=o, x index=1, y index=0] {results/threads-parallel_fastmath_numba.dat};
        \addplot table[color=red, mark=x, x index=1, y index=0] {results/threads-mp_python.dat};
        \legend{Cython, Cython Native, Numba, Numba Fastmath, MP Python}
    \end{axis}
\end{tikzpicture}

        \caption{Czas wykonania algorytmu w zależności od liczby wątków}
        \label{fig:all-threads}
    \end{minipage}
    \hfill
    \begin{minipage}[b]{.45\textwidth}
        \centering
        \begin{tikzpicture}[trim axis left]
    \begin{axis}[
        xlabel=Liczba wątków,
        ylabel=Przyspieszenie,
        legend pos=south east]
        \addplot table[color=red, mark=x, x index=1, y index=2] {results/threads-mp_cython.dat};
        \addplot table[color=blue, mark=o, x index=1, y index=2] {results/threads-mp_native_cython.dat};
        \addplot table[color=red, mark=x, x index=1, y index=2] {results/threads-parallel_numba.dat};
        \addplot table[color=blue, mark=o, x index=1, y index=2] {results/threads-parallel_fastmath_numba.dat};
        \addplot table[color=red, mark=x, x index=1, y index=2] {results/threads-mp_python.dat};
        \legend{Cython, Cython Native, Numba, Numba Fastmath, MP Python}

        \addplot[domain=0:9, samples=2, color=gray, style=dashed]{x};
    \end{axis}
\end{tikzpicture}

        \caption{Przyspieszenie liczenia liczby $\pi$}
        \label{fig:all-speedup}
    \end{minipage}
\end{figure*}


\section{Podsumowanie}

Na wykresie~\ref{fig:all-sizes-seq} widać porównanie
wszystkich wariantów sekwencyjnych.
Czysty Python oraz Cython są najmniej wydajne,
prawie 60 razy mniej wydajne niż wariant Numby.
Niemniej jednak Cython lekko zwiększa wydajność czystego Pythona.
Nieoczywistym jest jednak porównanie natywnego Cythona z~Numbą,
która wydaje się szybsza.
Okazuje się, że problem leży w~generowaniu liczb losowych ---
Numba ma dużo szybszy generator niż ten ze standardowej biblioteki
języka C\@.
Gdy porównywaliśmy czasy wykonania na własnym generatorze liczb pseudolosowych
ta różnica czasów wykonania się zacierała.
Natywny Cython był wtedy niewiele szybszy od Numby.

Wykres~\ref{fig:all-sizes-parallel} przedstawia porównanie wariantów równoległych.
Wykresy mają niemalże identyczne przebiegi jak te z~wariantu sekwencyjnego,
z~tą różnicą, że algorytmy równoległe są o~ponad pół rzędu wielkości szybsze
od swoich sekwencyjnych odpowiedników.
Np. czas wykonania programu
dla problemu o~rozmiarze 290000000 wersją sekwencyjną Numby wynosi 3.06s,
a~wersją równoległą 0.51s.
Powyżej opisana sytuacja z~generatorami liczb również jest tutaj widoczna.

Wykres~\ref{fig:all-threads} przedstawia czas wykonania problemu o rozmiarze
250000000 w zależności od liczby wątków,
przekłada się on na wykres przyspieszenia~\ref{fig:all-speedup}.
Wersje: czysty Python oraz Cython skalują się najlepiej ze wszystkich,
jednak należy wziąć pod uwagę, że czas wykonania w nich był najdłuższy.

Numba skalowała się relatywnie najgorzej.
Pięć wątków okazuje się najoptymalniejszym wyborem,
każda większa liczba nie daje wystarczająco dobrego przyspieszenia.
Ciężko spekulować dlaczego tak się wydarzyło, ze względu na to,
że cały kod zarządzający wątkami jest zaimplementowany w~Numbie
i~z~naszego punktu widzenia jest to czarna skrzynka.

Pod uwagę należy również wziąć parametry środowiska testowego.
Procesor, na którym wykonywane były testy posiada 8~fizycznych rdzeni.
Liczba 16 wątków na wykresach bierze się z opcji hyper-threading'u.
Z~tego wynika, że przynajmniej do 8 wątków programy powinny skalować się
niemalże zgodnie z~przewidywaniami teoretycznymi, a~ewentualne braki
dodatkowej wydajności powinny pojawić się dopiero od liczby 9~wątków.

Podsumowując, uważamy że Python ma rację bytu w HPC\@.
Istnieje w nim wiele możliwości na zwiększenie wydajności obliczeń,
a sam język może zostać wykorzystany do wysokopoziomowego opisu obliczeń.

Najlepiej do tego celu nadawałaby się Numba, gdyż wyróżniała się ona
wysoką, niemalże natywną wydajnością kodu.
Najlepsze wyniki uzyskał natomiast Cython Native,
ale pisanie w~nim tak naprawdę jest równoznaczne z~pisaniem w~języku C,
co w~żaden sposób nie wykorzystuje funkcji Pythona.

Kompilator Cython w~porównaniu do Numby i~interpretera Pythona daje zbyt
małe przyspieszenie, dlatego bardzo słabo nadaje się do wykonywania
krytycznych czasowo obliczeń.
Nadaje się jednak idealnie do integracji Pythona z językiem C\@.
Można go użyć na przykład aby wykorzystać w~Pythonie biblioteki napisane w~C,
lub innym języku, którego można zintegrować z~C~/~C++\@.


\section{Przyszłe prace}

Powyżej opisane wyniki niosą za sobą wiele kwestii wartych poruszenia.
Na pewno warto zgłębić problem skalowalności Numby na więcej niż 4--5
rdzeni.
Bardzo daleko jej było do perfekcyjnej skalowalności,
co może oznaczać że w~kodzie Numby może leżeć błąd,
ewentualnie objawia się on dla specyficznej architektury i~narzędzi.

Kolejną kwestią był nagły ujemny skok w~przyspieszeniu dla
liczby wątków równej 9 dla multiprocessingu.
Jest on ciekawą obserwacją, ponieważ w~aplikacjach wielowątkowych
napisanych w~innych językach niż Python on nie występuje ---
powyżej 8 wątków wydajność nadal rośnie.
Najpewniej jest on spowodowany specyfiką biblioteki multiprocessing
i~faktu, że wykorzystuje ona procesy zamiast wątków.
