\section{Podsumowanie}

Na wykresie~\ref{fig:all-sizes-seq} widać porównanie
wszystkich wariantów sekwencyjnych.
Czysty Python oraz Cython są najmniej wydajne,
prawie 60 razy mniej wydajne niż wariant Numby.
Niemniej jednak Cython lekko zwiększa wydajność czystego Pythona.
Nieoczywistym jest jednak porównanie natywnego Cythona z Numbą,
która wydaje się szybsza.
Okazuje się, że problem leży w generowaniu liczb losowych ---
Numba ma dużo szybszy generator niż ten ze standardowej biblioteki
języka C\@.
Gdy porównywaliśmy czasy wykonania na własnym generatorze liczb pseudolosowych
ta różnica czasów wykonania się zacierała.
Natywny Cython był wtedy szybszy od Numby.

Wykres~\ref{fig:all-sizes-parallel} przedstawia porównanie wariantów równoległych.
Wyniki są niemalże identyczne z wariantami sekwencyjnymi.
Sytuacja z generatorami liczb tutaj jest również widoczna.

\begin{figure*}
    \centering
    \begin{tikzpicture}[trim axis left]
    \begin{loglogaxis}[
        xlabel=Rozmiar problemu,
        ylabel=Czas wykonania,
        legend pos=outer north east]
        \addplot table[x index=1, y index=0] {results/sizes-seq_cython.dat};
        \addplot table[x index=1, y index=0] {results/sizes-seq_native_cython.dat};
        \addplot table[x index=1, y index=0] {results/sizes-seq_numba.dat};
        \addplot table[x index=1, y index=0] {results/sizes-seq_python.dat};
        \legend{Cython, Cython Native, Numba, Python, MP Python}
    \end{loglogaxis}
\end{tikzpicture}

    \caption{Czas wykonania algorytmu w zależności od rozmiaru problemu}
    \label{fig:all-sizes-seq}
\end{figure*}


\begin{figure*}
    \centering
    \begin{tikzpicture}[trim axis left]
    \begin{loglogaxis}[
        xlabel=Rozmiar problemu,
        ylabel=Czas wykonania,
        legend pos=outer north east]
        \addplot table[mark=*, x index=1, y index=0] {results/sizes-mp_cython.dat};
        \addplot table[mark=., x index=1, y index=0] {results/sizes-mp_native_cython.dat};
        \addplot table[mark=*, x index=1, y index=0] {results/sizes-parallel_numba.dat};
        \addplot table[mark=., x index=1, y index=0] {results/sizes-parallel_fastmath_numba.dat};
        \addplot table[mark=*, x index=1, y index=0] {results/sizes-mp_python.dat};

        \legend{MP Cython, MP Cython Native, Parallel Numba, Parallel Numba Fastmath, MP Python}
    \end{loglogaxis}
\end{tikzpicture}

    \caption{Czas wykonania algorytmu w zależności od rozmiaru problemu}
    \label{fig:all-sizes-parallel}
\end{figure*}

Wykres~\ref{fig:all-threads} przedstawia czas wykonania wystarczająco dużego problemu
w zależności od liczby wątków,
przekłada się on na wykres przyspieszenia~\ref{fig:all-speedup}.
Wersje: czysty Python oraz Cython skalują się najlepiej ze wszystkich,
jednak należy wziąć pod uwagę, że czas wykonania w nich był najdłuższy.
Skalowalność więc w innych przypadkach mogła się pogorszyć ze względu na to,
że kod równoległy się wykonywał szybciej, a co za tym idzie większa
część kodu sekwencyjnego wpływała na przyspieszenie.

Numba skalowała się relatywnie najgorzej.
Cztery wątki okazują się najoptymalniejszym wyborem,
każda większa liczba nie daje wystarczająco dobrego przyspieszenia.
Ciężko spekulować dlaczego tak się wydarzyło, ze względu na to,
że cały kod zarządzający wątkami jest zaimplementowany w Numbie
i z naszego punktu widzenia jest to czarna skrzynka.

\begin{figure*}
    \centering
    \begin{minipage}[b]{.45\textwidth}
        \centering
        \begin{tikzpicture}[trim axis left]
    \begin{axis}[
        xlabel=Liczba wątków,
        ylabel=Czas wykonania,
        legend pos=north east]
        \addplot table[color=red, mark=x, x index=1, y index=0] {results/threads-mp_cython.dat};
        \addplot table[color=blue, mark=o, x index=1, y index=0] {results/threads-mp_native_cython.dat};
        \addplot table[color=red, mark=x, x index=1, y index=0] {results/threads-parallel_numba.dat};
        \addplot table[color=blue, mark=o, x index=1, y index=0] {results/threads-parallel_fastmath_numba.dat};
        \addplot table[color=red, mark=x, x index=1, y index=0] {results/threads-mp_python.dat};
        \legend{Cython, Cython Native, Numba, Numba Fastmath, MP Python}
    \end{axis}
\end{tikzpicture}

        \caption{Czas wykonania algorytmu w zależności od liczby wątków}
        \label{fig:all-threads}
    \end{minipage}
    \hfill
    \begin{minipage}[b]{.45\textwidth}
        \centering
        \begin{tikzpicture}[trim axis left]
    \begin{axis}[
        xlabel=Liczba wątków,
        ylabel=Przyspieszenie,
        legend pos=south east]
        \addplot table[color=red, mark=x, x index=1, y index=2] {results/threads-mp_cython.dat};
        \addplot table[color=blue, mark=o, x index=1, y index=2] {results/threads-mp_native_cython.dat};
        \addplot table[color=red, mark=x, x index=1, y index=2] {results/threads-parallel_numba.dat};
        \addplot table[color=blue, mark=o, x index=1, y index=2] {results/threads-parallel_fastmath_numba.dat};
        \addplot table[color=red, mark=x, x index=1, y index=2] {results/threads-mp_python.dat};
        \legend{Cython, Cython Native, Numba, Numba Fastmath, MP Python}

        \addplot[domain=0:9, samples=2, color=gray, style=dashed]{x};
    \end{axis}
\end{tikzpicture}

        \caption{Przyspieszenie liczenia liczby $\pi$}
        \label{fig:all-speedup}
    \end{minipage}
\end{figure*}
