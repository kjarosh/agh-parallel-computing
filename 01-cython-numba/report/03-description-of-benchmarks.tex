\section{Opis benchmarków}

\subsection{Przybliżanie liczby $\pi$}

Pierwszym benchmarkiem jest przybliżanie liczby $\pi$ za pomocą
algorytmu Monte Carlo.
Losujemy zadaną liczbę punktów z kwadratu jednostkowego
i obliczamy jaka część należy do ćwiartki koła.
Algorytm ten jest bardzo łatwy w zrównolegleniu
i pozwala na dobrą reprezentację wyników.

Wadą tego algorytmu jest natomiast ekstensywne wykorzystywanie
generatora liczb pseudolosowych.
W naszym przypadku, gdy korzystamy z wielu narzędzi, może
to oznaczać duże wahania w wynikach w zależności od użytego narzędzia.
Jak się później okaże, różne generatory mogą dawać istotnie różne wyniki,
czasem sięgające dwóch rzędów wielkości.
Problem ten ukazał się już po wykonaniu eksperymentów,
podczas interpretacji wyników.

\subsection{Mnożenie wektorów}

Kolejnym benchmarkiem jest mnożenie dwóch wektorów o rozmiarach
$n$ tak, aby otrzymać macierz rozmiaru $n\times n$.
Wybraliśmy taki test głównie po to, aby porównać wydajność
z biblioteką NumPy bez niepotrzebnej implementacji skomplikowanych
algorytmów.
