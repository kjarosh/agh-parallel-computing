\section{Numba}
Obliczenia przeprowadzane w Numbie oprócz podziału na algorytm sekwencyjny i równoległy
zostały również podzielone na te z włączoną i wyłączoną  opcją `Fastmath`.
Rozmiar problemu dla algorytmu równoległego był równy 1000000 tzn., że do obliczenia
liczby Pi losowaliśmy 1000000 punktów.
Ze względu na to, że żadna stabilna wersja Numby nie umożliwiała ręcznego ustawienia
liczby wątków musieliśmy skorzystać z wersji, która jest jeszcze w fazie testów.


\subsection{Przybliżanie liczby pi}

\begin{figure*}
    \centering
    \begin{minipage}[b]{.45\textwidth}
        \centering
        \begin{tikzpicture}[trim axis left]
    \begin{axis}[
        xlabel=Liczba wątków,
        ylabel=Przyspieszenie,
        legend pos=north west]
        \addplot table[x index=1, y index=2] {results/threads-parallel_numba.dat};
        \addplot table[x index=1, y index=2] {results/threads-parallel_fastmath_numba.dat};
        \addplot table[x index=1, y index=2] {results/threads-mp_numba.dat};
        \addplot table[x index=1, y index=2] {results/threads-mp_fastmath_numba.dat};

        \addplot[domain=0:9, samples=2, color=gray, style=dashed]{x};

        \legend{Numba, Numba Fastmath, MP Numba, MP Numba Fastmath}
    \end{axis}
\end{tikzpicture}

        \caption{Przyspieszenie liczenia liczby $\pi$ w Numbie}
        \label{fig:numba-speedup}
    \end{minipage}
    \hfill
    \begin{minipage}[b]{.45\textwidth}
        \centering
        \begin{tikzpicture}[trim axis left]
    \begin{axis}[
        xlabel=Liczba wątków,
        ylabel=Czas wykonania,
        legend pos=north east]
        \addplot table[color=red, mark=x, x index=1, y index=0] {results/threads-parallel_numba.dat};
        \addplot table[color=blue, mark=o, x index=1, y index=0] {results/threads-parallel_fastmath_numba.dat};

        \legend{Numba, Numba Fastmath}
    \end{axis}
\end{tikzpicture}

        \caption{Czas wykonania algorytmu w Numbie w zależności od liczby wątków}
        \label{fig:numba-threads}
    \end{minipage}
\end{figure*}

\begin{figure*}
    \centering
    \begin{tikzpicture}[trim axis left]
    \begin{axis}[
        xlabel=Rozmiar problemu,
        ylabel=Czas wykonania (s),
        legend pos=north west]
        \addplot table[mark=*, x index=1, y index=0] {results/sizes-parallel_numba.dat};
        \addplot table[mark=., x index=1, y index=0] {results/sizes-parallel_fastmath_numba.dat};

        \legend{Parallel Numba, Parallel Numba Fastmath}
    \end{axis}
\end{tikzpicture}

    \caption{Czas równoległego wykonania algorytmu w Numbie w zależności od rozmiaru problemu}
    \label{fig:numba-parallel-sizes}
\end{figure*}

Wykres ~\ref{fig:numba-threads} przedstawia zależność czasu rozwiązania problemu
od ilości wykorzystywanych
wątków.
Możemy zaobserwować, że wraz ze zwiększaniem liczby wątków istotnie maleje czas
wykonania programu.
Program skaluje się bardzo dobrze dla 2, 3 i 4 wątków natomiast
dokładanie kolejnych tj. 5, 6, 7, 8 nie daje już aż tak dobrego rezultatu.
Widać to dokładnie
na wykresie ~\ref{fig:numba-speedup} gdzie począwszy od 5 wątków realne przyspieszenie
programu zdecydowanie odbiega od teoretycznego.

%Ostatni wykres ~\ref{fig:numba-parallel-sizes}) przedstawia czas równoległego wykonania
%programu w Numbie w zależności od rozmiaru problemu. Wyniki są zgodne z oczekiwanymi.
%Dla problemu o małym rozmiarze (liczba losowanych punktów = 100000) algorytm sekwencyjny
%jest szybszy od algorytmu równoległego, natomiast już od rozmiaru równego
%200000 sytuacja się odwraca. Dla obu opcji zostały również wykonane pomiary z opcją
%fastmath, która minimalnie przyspiesza działanie programów.



%\subsection{Mnożenie wektorów}
