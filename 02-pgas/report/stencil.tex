\section{Stencil (Kamil Jarosz)}

\begin{table}[]
    \centering
    \begin{tabular}{|r|c|c|c|c|c|c|c|c|}
        \hline
        element & 1 & 2 & 3 & 4 & 5 & 6 & 7 & 8          \\ \hline
        proces & \multicolumn{2}{c|}{P1} & \multicolumn{2}{c|}{P2} & \multicolumn{2}{c|}{P3} & \multicolumn{2}{c|}{P4} \\ \hline
    \end{tabular}
    \caption{Przykładowe rozlokowanie w pamięci tablicy stencila}
    \label{fig:stencil-table}
\end{table}

Benchmark `stencil', zaproponowany w dokumentacji UPC++ polega
na operowaniu na pewnej rozdystrybuowanej tablicy.
Wykonywane są na niej operacje, które pozwalają ustalić końcowy jej stan.
W tym konkretnym przypadku jest to liczenie średniej wartości elementów tablicy
-- każdy element w pojedynczej iteracji staje się sumą sąsiadów
(równanie~\ref{eq:stencil-iter}).

\begin{gather}
    u_i = \frac{u_{i-1}+u_{i+1}}{2}
    \label{eq:stencil-iter}
\end{gather}

Rysunek~\ref{fig:stencil-table} reprezentuje przykładowe ulokowanie
12-elementowej tablicy stencila w modelu pamięci PGAS\@.
Każdy proces jest właścicielem części pamięci.
W przypadku liczenia średniej wartości, procesy sąsiednie muszą
komunikować się i wymieniać informacjami na temat elementów na krawędzi.
Dodatkowo wprowadzone zostały dwie fazy liczenia średniej -- dla elementów
parzystych i nieparzystych, po to aby race conditions nie występowały.

\begin{gather}
    \text{err} = \varepsilon \cdot E
    \label{eq:stencil-conv}
\end{gather}

Co stałą liczbę iteracji sprawdzany jest warunek zbieżności.
W tym przypadku jest to równanie~\ref{eq:stencil-conv},
gdzie $\text{err}$ jest maksymalną różnicą między wartościami
elementów a wartością oczekiwaną, $\varepsilon$ jest dobraną odpowiednio małą
stałą, natomiast $E$ jest wartością oczekiwaną.

Benchmark ten pozwala sprawdzić wiele przydatnych rzeczy na temat danego narzędzia:
dostępność konstrukcji pozwalających w optymalny sposób zaimplementować go lub
wydajność komunikacji i synchronizacji między procesami.
Jest on zorientowany na używanie barier, oraz redukcję wyników
z wielu procesów.

\subsection{Przygotowanie środowiska}

Chapel na pierwszy rzut oka jest bardzo łatwym w użyciu językiem.
Banalnie można go skompilować i uruchomić.
Chapel zadziała od razu, jednak nie od razu będzie wydajny.
Wbrew pozorom bardzo trudno jest napisać program w Chapelu,
który będzie szybko działać.

Pierwszą rzeczą jaka może mieć wpływ na porównanie Chapela i UPC++
jest komunikacja między procesami.
Zarówno Chapel jak i UPC++ pod spodem używają biblioteki GASNet~\cite{gasnet},
jednak ich domyślne ustawienia się diametralnie różnią.
Domyślnie Chapel używa warstwy komunikacji opartej na UDP, gdy
UPC++ używa pamięci współdzielonej.
Stąd wynika wymóg ustawienia zmiennej środowiskowej
\texttt{CHPL\_COMM\_SUBSTRATE=smp}.
Chapel jednak nie jest domyślnie dystrybuowany ze skompilowaną warstwą \texttt{smp},
stąd wymóg jego rekompilacji.

Kolejnym istotnym faktem jest optymalizacja pod konkretną platformę.
C++ tworzy kod zoptymalizowany pod daną platformę, gdy Chapel domyślnie
kompiluje do kodu niezoptymalizowanego.
Stąd wymóg ustawienia zmiennej \texttt{CHPL\_TARGET\_CPU=native}.
Wymaga to również przekompilowania Chapela
oraz pozwoli na użycie flagi kompilatora \texttt{--fast}.

Kolejną ważną różnicą jest fakt braku wielowątkowości w obrębie
jednego procesu w UPC++.
Aby Chapel nie wykorzystywał więcej niż jednego wątku na danym locale,
ustawiana jest zmienna procesowa \texttt{CHPL\_RT\_NUM\_THREADS\_PER\_LOCALE=1}.
Dodatkowo w kodzie nie są wykorzystywane instrukcje korzystające z
wielu wątków.

Testy zostały przeprowadzone na komputerze wyposażonym w procesor
Intel Core i5-9600K CPU @ 3.70GHz (6 rdzeni, 9MB cache, turbo)
oraz pamięć operacyjną 16GB, 3200 MHz, CL16.
Były przeprowadzone z wykorzystaniem systemu Linux Mint 19.3,
oraz z wersją Chapela 1.22.0 (używaną na Dockerze).
UPC++ było używane w wersji 2020.3.0 wraz
z gcc 7.5.0.


\subsection{Implementacja benchmarku}

Ustalane są pewne stałe:
\begin{itemize}
    \item $n=1024\cdot16$ --- rozmiar stencila,
    \item $M=100$ --- maksymalna wartość losowanych liczb,
    \item $\varepsilon=0.1$ --- wspomniana stała z równania~\ref{eq:stencil-conv},
    \item $E=M/2$ --- oczekiwana średnia.
\end{itemize}

Na postawie $n$ losowana jest tablica z rozkładem jednostajnym
z przedziału $[0,M]$.
Jest ona rozproszona na różnych procesach zgodnie ze schematem
z rysuknu~\ref{fig:stencil-table}.

Następnie wykonywane są iteracje, każda iteracja w zależności od fazy
(parzyste iteracje są wykonywane w fazie 0, nieparzyste w fazie 1)
aktualizuje parzyste (faza 0) lub nieparzyste (faza 1) elementy.

Co 200 iteracji sprawdzana jest zbieżność.
Każdy proces liczy maksymalną absolutną różnicę między elementami
a wartością oczekiwaną, a następnie dane są zbierane z wszystkich procesów
i finalny błąd $\text{err}$ jest liczony jako ich wartość maksymalna.
Następnie aplikowany jest warunek zbieżności (równanie~\ref{eq:stencil-conv}).

Stała 200 nie została wybrana przypadkowo --- UPC++ ma wbudowane
algorytmy wykonujące redukcję na wielu procesach, które są bardzo
zoptymalizowane.
Chapel niestety nie ma takich konstrukcji, ani funkcji bibliotecznych
co dla częstego sprawdzania warunku zbieżności
powodowało, że czas wykonania rósł
niemalże liniowo względem liczby procesów:
dla dwóch procesów średnio wykonywał się dwa razy \textit{wolniej}
niż dla jednego.
Wynikało to z problemu bardzo częstej komunikacji między procesami.

\subsection{Porównanie}

Finalny kod w Chapelu był dużo bardziej czytelny
oraz zwięzły niż w UPC++.
Chapel pozwala na mapowanie domen, co spowodowało że
rozproszenie tablicy na wiele procesów było opisane jedną
instrukcją, a użycie rozproszonej tablicy było identyczne
z użyciem zwykłej.
Chapel automatycznie komunikuje się z innymi procesami gdy
potrzebujemy dostępu do pamięci z innego procesu.
Niewątpliwą zaletą jest łatwość użycia, jednak
istotną wadą jest dużo potencjalnych pomyłek, które nie sprawią
że program będzie działać niepoprawnie,
a że będzie on bardzo niewydajny.
Dlatego pomocny był moduł \texttt{CommDiagnostics}, który
pozwalał na logowanie komunikacji między procesami.
Dzięki temu można być pewnym, że przesyłane są tylko
dane które powinny być przesyłane.

W UPC++ rozłożenie pamięci między procesy jest częścią struktury kodu.
Gdy stwierdzimy, że jednak chcemy inne rozlokowanie, w Chapelu jest to
jedna linijka zmiany, natomiast w UPC++ oznacza to przepisanie sporego
fragmentu kodu.
Utrudnia to jednak pisanie niewydajnego kodu.
Każda komunikacja między procesami musi być napisana explicite.
W Chapelu wszystko dzieje się automatycznie.

W przypadku obu narzędzi barier używa się bardzo podobnie.
W UPC++ jest to funkcja \texttt{upcxx::barrier()},
w Chapelu jest to funkcja \texttt{allLocalesBarrier.barrier()}.
Bariery w UPC++ były dużo wydajniejsze niż w Chapelu.

Kolejnym istotnym elementem są redukcje.
W tym benchmarku często pojawia się sytuacja, gdy każdy proces
dysponuje pewną wartością i należy ją zredukować do jednej, wspólnej.
W UPC++ nie ma z tym problemu, gdyż funkcja \texttt{upcxx::reduce\_all}
dokładnie do tego służy.
Jest ona bardzo wydajna i każdemu procesowi zwraca zredukowany wynik.
W Chapelu niestety nie istnieje taki mechanizm.
Dysponuje on konstrukcją \texttt{reduce}, jednak nie pozwoli ona zredukować
wartości z wszystkich procesów.
Generalnie możliwe jest jej użycie w dwóch trybach:
\begin{itemize}
    \item na jednym procesie jest w stanie zredukować wartości
    kolekcji, używając wielu wątków --- to niestety nie poprawia sytuacji,
    gdyż na jednym procesie założyliśmy maksymalnie jeden wątek,
    \item jako redukcja wykonania z pętli \texttt{coforall}
    lub \texttt{forall} --- to rozwiązanie też nie jest optymalne, gdyż nie
    da się go użyć z wewnątrz pętli, tylko po jej wykonaniu.
\end{itemize}
Finalnie zostało to zaimplementowane w taki sposób, że
wybrany proces (zerowy) liczy maksymalną wartość z wartości podanych
przez wszystkie procesy, a następnie zwraca informację czy warunek był
prawdziwy.
Informacja jest zwracana za pomocą zmiennych synchronicznych,
czyli takich które udostępniają metody synchronizacji pomiędzy procesami.

Podczas testowania bardzo wyraźnie można było zobaczyć róznicę
w czasie uruchamniania programów.
W UPC++ program był uruchamiany natychmiastowo, gdy
w Chapelu na począku zawsze występowało bardzo zauważalne opóźnienie.
Stąd był wymóg liczenia czasu wykonania w kodzie Chapela,
w przeciwieństwie do UPC++, gdzie można było liczyć czas wykonania
całego procesu.


\section{Wyniki benchmarku}

\begin{figure}
    \centering
    \begin{tikzpicture}
    \begin{axis}[
        xlabel=Liczba procesów/locali,
        ylabel=Czas wykonania (s),
        width=\columnwidth,
        legend style={at={(0.97,0.4)},anchor=east}]
        \addplot table[x index=0, y index=2] {results/chapel-stencil.dat};
        \addplot table[x index=0, y index=2] {results/upcxx-stencil.dat};
        \legend{Chapel, UPC++}
    \end{axis}
\end{tikzpicture}

    \caption{Wykres czasu wykonania w zależności od liczby procesów/locali}
    \label{fig:stencil-times}
\end{figure}

\begin{figure}
    \centering
    \begin{tikzpicture}
    \begin{axis}[
        xlabel=Liczba procesów/locali,
        ylabel=Przyspieszenie,
        width=\columnwidth,
        legend pos=north west]
        \addplot table[x index=0, y index=3] {results/chapel-stencil.dat};
        \addplot table[x index=0, y index=3] {results/upcxx-stencil.dat};

        \addplot[domain=0:6, samples=2, color=gray, style=dashed]{x};

        \legend{Chapel, UPC++}
    \end{axis}
\end{tikzpicture}

    \caption{Wykres przyspieszenia w zależności od liczby procesów/locali}
    \label{fig:stencil-speedup}
\end{figure}

\begin{figure}
    \centering
    \begin{tikzpicture}
    \begin{loglogaxis}[
        xlabel=Rozmiar problemu,
        ylabel=Czas wykonania (s),
        width=\columnwidth,
        legend pos=north west]
        \addplot table[x index=0, y index=1] {results/stencil-seq.dat};
        \addplot table[x index=0, y index=2] {results/stencil-seq.dat};
        \legend{Chapel, UPC++}

        % \addplot[domain=0.2:6, samples=100, color=gray, style=dashed] {121.305 * x - 15.1874};
        % \addplot[domain=0.2:6, samples=100, color=gray, style=dashed] {13.72 * x - 4.93};
    \end{loglogaxis}
\end{tikzpicture}

    \caption{Wykres czasu wykonania w zależności od rozmiaru problemu}
    \label{fig:stencil-seq}
\end{figure}

Bardzo trudno było uzyskać wiarygodne wyniki benchmarku ze względu na
często niewytłumaczalny niedeterminizm Chapela.
Czasy wykonania były bardzo zróżnicowane pomiędzy różnymi środowiskami
a nawet uruchomieniami programu.
Bardzo często kod na wielu procesach wykonywał się tak samo długo jak na jednym.
Z tego powodu wybrane zostało środowisko, na którym czasy wykonania były
najbardziej stabilne.
Czasy wykonania były liczone jako minimum z 5 prób.

Z UPC++ nie było większych problemów.
Czasy wykonania były za każdym razem niemalże identyczne
i skalowały się dużo lepiej.

Testy zależne od liczby procesów/locali zostały przeprowadzone na stałym
rozmiarze problemu, który wynosił $1024 \cdot 45 \cdot 8$ (rozmiar tablicy stencila).
Wykorzystany epsilon był równy $0.1$.

Wykres~\ref{fig:stencil-times} przedstawia czas wykonania w zależności od liczby
procesów/locali.
UPC++ charakteryzowało się dużo niższym czasem wykonania programu niż
Chapel.
Był on 6 razy szybszy dla wariantu na jednym procesie/localu,
i ponad 15 razy szybszy dla 5 procesów/locali.

Rysunek~\ref{fig:stencil-speedup} przedstawia przyspieszenie w obliczeniach
w zależności od liczby procesów/locali.
W UPC++ przyspieszenie się zwiększa aż do 4 procesów.
Dla 5 pozostaje na stałym poziomie, a dla 6 spada.
W przypadku Chapela sytuacja jest katastrofalna.
Dla więcej niż 2 locali przyspieszenie nie rośnie, lecz stale maleje.

Na wykresie~\ref{fig:stencil-seq} przedstawiona jest zależność czasu
wykonania od rozmiaru problemu.
Rozmiar problemu podany jest jako współczynnik przed bazowym rozmiarem
problemu używanym w powyższych testach.

\subsection{Wnioski}

Wyniki jednoznacznie pokazują, że Chapel skaluje się dużo gorzej i
ogólny czas jego wykonania jest zauważalnie większy.

Bardzo ciężko powiedzieć skąd wynikał tak duży rozrzut czasów wykonania
programu w przypadku Chapela.
Przyczyną może być fakt, że Chapel był uruchomiony na wielu
locale'ach, ale na jednej maszynie.
O ile UPC++ jest dostosowany do takiego sposobu uruchomienia,
o tyle Chapel sugeruje aby jako jeden locale traktować jedną maszynę,
co może sugerować że nie jest on zoptymalizowany pod taki układ.
Chapel jest również językiem używanym głównie na maszynach Cray,
więc zapewne to właśnie pod nie będzie on najbardziej dostosowany.

Wpływ na czasy wykonania na pewno miał też sam problem.
Chapelowi brakowało odpowiednich metod redukcji wartości z wielu locali.
To z pewnością wpłynęło negatywnie na czas wykonania gdyż zwiększało
intensywność komunikacji i czas oczekiwania na wyniki.

\subsection{Przyszłe prace}

Niestety w tym sprawozdaniu już zabrakło czasu,
ale prostym i wartym przeprowadzenia eksperymentem jest
użycie opcji \texttt{--llvm}.
Jest ona eksperymentalną funckją kompilatora Chapela i pozwala na
generację kodu do LLVM\@.
Może to pozwolić na lepsze optymalizacje i drastycznie zmienić wyniki
eksperymentu.

Kolejnym pomysłem jest uruchomienie benchmarku używając innego
sposobu komunikacji niż pamięć współdzielona (na przykład sieć)
na wielu maszynach -- tak jak sugeruje dokumentacja.

Ciekawą opcją byłoby również uruchomienie kodu na systemach Cray
i porównaniu na tychże systemach wydajności programów.
Taki test byłby trudniejszy
z powodu braku dostępności takich systemów.
