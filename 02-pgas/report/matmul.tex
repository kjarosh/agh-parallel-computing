\section{Mnożenie macierzy (Wiktor Sus)}
    Zadanie polega na wykonaniu mnożenia macierzowego dla dwóch kwadratowych macierzy o zadanym rozmiarze.
    W celu zachowania spójności rezultatów, macierze wejściowe są inicjalizowane z wykorzystaniem generatora liczb losowych z zadanym ziarnem.
    
    \subsection{Chapel}
    Początkowo przeprowadzone zostały testy dla języka Chapel w konfiguracji wykorzystującej do komunikacji protokół udp.
    Jest to typowa konfiguracja dla wielu jednostek obliczeniowych.
    Obliczenia wykonywane były dla różnych rozmiarów problemów i ilości lokali (jednostek obliczeniowych) oraz były powtarzane dwa razy, wynikiem jest średnia obu czasów wykonania.
    Prędko okazało się jednak, że zwiększenie ilości lokali dla problemów o rozmiarze większym niż 100 (od 200 do 1000) powodowało niemal wykładnicze zwiększenie czasu obliczeń, do stopnia w którym rozwiązanie problemu na jednym lokalu trwało ~1s, lecz na dwóch były to już 3min.
    Z tego względu nie zostało wykonane pełne sprawozdanie dla różnych rozmiarów problemów.
    
    Kolejną metodą komunikacji jest smp (pamięć współdzielona).
    W tym przypadku czas obliczeń znacząco zmalał, jednak ponownie zwiększenie liczby jednostek powodowało wzrost czasu wykonania.
    W tabeli \ref{tab:matrixmulchapelcomm} zebrane zostały czasy wykonania w zależności od ilości lokali dla obu protokołów.
    Widać w niej wyraźne przyspieszenie względem protokołu udp, jednak skalowanie się czasu obliczeń w zależności od ilości jednostek jest odwrotne niż oczekiwane.
    
    \begin{table}[]
    \centering
    \begin{tabular}{l|l|l}
    \textbf{lokale} & \textbf{udp} & \textbf{smp} \\ \hline
    1                & 1.09443s     & 1.0625s      \\
    2                & 7.05248s     & 1.71718s     \\
    3                & 10.0483s     & 1.91891s     \\
    4                & 18.1122s     & 2.37569s     \\
    5                & 16.0015s     & 2.15469s     \\
    6                & 24.9213s     & 3.37137s     \\
    7                & 21.0581s     & 4.246s       \\
    8                & 26.7952s     & 5.12254s    
    \end{tabular}
    \caption{Porównanie obliczeń przy użyciu smp i udp w języku Chapel dla stałego ziarna oraz problemu o rozmiarze 100.}
    \label{tab:matrixmulchapelcomm}
    \end{table}
    
    Ze względu na krótszy czas wykonania obliczeń dla protokołu smp, przeprowadzono testy dla większych problemów, których wyniki zebrane zostały w tabeli \ref{tab:matrixmulchapelsmp}.
    Ponownie zwiększenie ilości jednostek nie poprawiło czasu wykonania programu.
    
    \begin{table}[]
    \centering
    \begin{tabular}{l|l|l|l}
    \textbf{lokale} & \textbf{100} & \textbf{200} & \textbf{500} \\ \hline
    1               & 1.05837s     & 8.37514s     & 131.969s     \\
    2               & 1.68456s     & 13.4264s     & 209.775s     \\
    3               & 1.8937s      & 14.8816s     & 236.876s     \\
    4               & 2.31936s     & 18.3884s     & 286.015s     \\
    5               & 2.07s        & 19.0784s     & 259.122s     \\
    6               & 2.61636s     & 22.35s       & 340.724s     \\
    7               & 4.18683s     & 31.074s      & 378.78s      \\
    8               & 5.02199s     & 30.7993s     & 529.622s    
    \end{tabular}
    \caption{Porównanie czasów obliczeń w Chapel przy użyciu smp dla różnych rozmiarów problemów.}
    \label{tab:matrixmulchapelsmp}
    \end{table}
    
    \subsection{UPC++}
    Drugim testowanym rozwiązaniem była biblioteka UPC++.
    Implementuje ona model PGAS dla języka C++ i pozwala na względnie łatwe programowanie wysokowydajnościowych obliczeń.
    Testy przeprowadzone zostały dla tego samego problemu, w konfiguracji wykorzystującej od 1 do 8 procesów (jednostek obliczeniowych) korzystając z komunikacji poprzez pamięć współdzieloną (smp), bez powtarzania poszczególnych obliczeń.
    Wyniki przedstawione zostały w tabeli \ref{tab:matrixmulupc}.
    Możemy zauważyć przyspieszenie wykonania programu wraz z zwiększaniem ilości rdzeni, choć zwiększanie liczby rdzeni powyżej 3 daje coraz mniej zauważalne rezultaty.
    Jest to najprawdopodobniej spowodowane początkowym kopiowaniem tablicy do pamięci każdego procesu, co wraz ze wzrostem ich ilości naturalnie wydłuża całkowity czas działania programu.
    Rozmiar problemu ograniczony jest zasobami sprzętowymi komputera, w przypadku jednostki posiadającej 16GB pamięci operacyjnej obliczenie problemu o rozmiarze powyżej 2000 jest niewykonalne ze względu na brak wystarczającej ilości pamięci (trzy tablice kwadratowe o wymiarze ponad 2000, zawierające liczby podwójnej precyzji).
    
    \begin{table}[]
    \centering
    \begin{tabular}{l|l|l|l}
    \textbf{lokale} & \textbf{1000} & \textbf{1500} & \textbf{2000} \\ \hline
    1               & 1.84154s      & 6.64677s      & 29.9616s      \\
    2               & 1.12929s      & 4.10752s      & 16.3781s      \\
    3               & 0.888757s     & 2.89905s      & 11.3491s      \\
    4               & 1.92441s      & 2.49104s      & 16.5829s      \\
    5               & 1.79313s      & 6.37188s      & 15.3325s      \\
    6               & 1.58884s      & 5.2607s       & 13.3494s      \\
    7               & 1.42547s      & 4.01053s      & 10.5655s      \\
    8               & 1.85513s      & 4.35211s      & 11.1056s     
    \end{tabular}
    \caption{Porównanie czasów obliczeń w UPC dla różnych rozmiarów problemów.}
    \label{tab:matrixmulupc}
    \end{table}
    
    \subsection{Podsumowanie}
    Jak łatwo można zauważyć, implementacja zadania przy użyciu UPC++ jest o wiele wydajniejsza od wersji w języku Chapel.
    Może to wynikać z niewystarczającej znajomości języka lub błędów programisty.
    Przy wyborze narzędzia do obliczeń w modelu PGAS warto wziąć pod uwagę fakt, że UPC++ opiera się o język C++, dostarczając jedynie biblioteki, która jest dobrze udokumentowana oraz bez większych problemów spełnia swoje zadanie w satysfakcjonującym stopniu.
    Nie można tego samego powiedzieć o języku Chapel, który opisany jest jedynie na oficjalnej stronie.
    Brakuje dla niego wyczerpujących przykładów oraz informacji jak poprawnie w nim programować, tak by uzyskać oczekiwane skalowanie się rozwiązania.
