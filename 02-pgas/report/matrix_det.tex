\section{Wyznacznik macierzy (Michał Śledź)}
\label{sec:matrix_det}

\subsection{Środowisko testowe}

\begin{description}
    \item [procesor]: i5-9600K 6 rdzeni, 3,70-4,60 GHz, 9MB Cache
    \item [pamięć]: 16 GB RAM, 3200 MHz, CL 16
\end{description}
Testy zostały wykonane na jednej maszynie lokalnej (opisanej powyżej) bez udziału
sieci.
Testy były uruchamiane wielokrotnie i dawały zbliżone do siebie wyniki.

\subsection{UPC++}

Na początku deklarujemy dwa distributed objecty, dzięki czemu
każdy z procesów będzie miał te same zmienne lecz z innymi
wartościami~\ref{lst:upcxx-dstobj-init}.

\begin{lstlisting}[
    caption=Inicjalizacja distributed objectów,
    label={lst:upcxx-dstobj-init}]
upcxx::dist_object<upcxx::global_ptr<double>>
    u_g(upcxx::new_array<double>(N*N));
upcxx::dist_object<upcxx::global_ptr<double>>
    sum(upcxx::new_<double>(0));
\end{lstlisting}

Następnie proces o id równym 0 inicjalizuje tablicę, dla której
mamy policzyć wyznacznik, a kolejne procesy kopiują
jej zawartość~\ref{lst:upcxx-state-sync}.

\begin{lstlisting}[
    caption=Ustalenie stanu procesów,
    label={lst:upcxx-state-sync}]
if (upcxx::rank_me() == 0) {
    init_matrix(&u_g);
}
upcxx::barrier();

upcxx::global_ptr<double> u =
    u_g.fetch(0).wait();
double **matrix = allocate_matrix(N);
for (int i = 0; i < N; i++) {
    for (int j = 0; j < N; j++) {
        int ptr = u + (i*N + j);
        matrix[i][j]=upcxx::rget(ptr).wait();
    }
}
\end{lstlisting}

Kolejnym krokiem jest obliczenie przez każdy z procesów
rozwinięcia Laplace'a, które oznaczone jest przez wywołanie metody run.
Każdy z procesów liczy wspomniane rozwinięcie tylko dla wybranych kolumn, zgodnie
ze wzorem:
\begin{equation}
    k = proc\_id + n*proc\_num
\end{equation}
Wywołujemy również funkcję \texttt{upcxx::barrier()} celem synchronizacji wszystkich
procesów w tym miejscu~\ref{lst:upcxx-laplace}.

\begin{lstlisting}[
    caption=Liczenie rozwinięcia Laplace'a,
    label={lst:upcxx-laplace}]
    double *local_sum = sum->local();
    *local_sum = run(matrix, N);
    upcxx::barrier();
\end{lstlisting}

Na końcu proces o id równym zero odczytuje zmienne sum pozostałych
procesów wyliczając w ten sposób ostateczną
wartość wyznacznika~\ref{lst:upcxx-sum-results}.

\begin{lstlisting}[
    caption=Sumowanie wyniku,
    label={lst:upcxx-sum-results}]
if (upcxx::rank_me() == 0) {
    int proc_n = upcxx::rank_n();
    for (int i = 1; i < proc_n; i++) {
        upcxx::global_ptr<double> u =
            sum.fetch(i).wait()
        *local_sum +=
            upcxx::rget(u).wait();
    }
    cout << "\ndet: " << *local_sum;
}
\end{lstlisting}

Wykres~\ref{fig:upcxx-matrixdet-time} przedstawia czas wykonania algorytmu w zależności
od liczby wykorzystanych wątków.
Rezultaty są spodziewane.
Dla 5 wątków nie uzyskujemy żadnego przyspieszenia w stosunku
do 4 co jest spowodowane tym, że rozmiar testowej macierzy nie jest podzielny przez 5.
W rezultacie w ostatniej iteracji trzy wątki będą bezczynne ale wciąż trzeba poczekać na
pozostałe dwa.
Stąd brak przyspieszenia.
Jest to dokładnie widoczne na wykresie~\ref{fig:upcxx-matrixdet-speedup}.

\begin{figure}[h]
    \centering
    \begin{tikzpicture}
    \begin{axis}[
        xlabel=Liczba wątków,
        ylabel=Czas wykonania (s),
        legend pos=north east]
        \addplot table[mark=*, x index=0, y index=1] {results/upcxx-matrixdet.dat};

        \legend{Czas wykonania}
    \end{axis}
\end{tikzpicture}

    \caption{Czas wykonania algorytmu w zależności od liczby wątków}
    \label{fig:upcxx-matrixdet-time}
\end{figure}

\begin{figure}[h]
    \centering
    \begin{tikzpicture}
    \begin{axis}[
        xlabel=Liczba wątków,
        ylabel=Przyspieszenie,
        width=\columnwidth,
        legend pos=south east]
        \addplot table[mark=x, x index=0, y index=2] {results/upcxx-matrixdet.dat};
        \addplot[domain=0:9, samples=2, color=gray, style=dashed]{x};
    \end{axis}
\end{tikzpicture}

    \caption{Przyspieszenie algorytmu w zależności od liczby wątków}
    \label{fig:upcxx-matrixdet-speedup}
\end{figure}

\subsection{Chapel}

Kod napisany w Chapelu jest analogiczny do tego napisanego w UPC++.
Główną różnicą jest brak bezpośrednich odniesień do pamięci zdalnej tzn.
nie wskazujemy explicite, że odwołujemy się do pamięci, która nie należy
do obecnego procesu.
Chapel robi to sam, za nas.

Reszta kodu jest bardzo intuicyjna i podobna do swojego odpowiednika w C++ dlatego
jego szczegółowa analiza nie ma sensu.
Ciekawe i zastanawiające są natomiast wyniki.


Wykres~\ref{fig:chapel-matrixdet-time} przedstawia czas wykonania programu w zależności
od liczby wątków.
Widzimy, że wykres jest daleki od oczekiwanego.
Przede wszystkim dla 3 oraz 5 wątków nie uzyskujemy przyspieszenia względem ich poprzedników
(wykres~\ref{fig:chapel-matrixdet-speedup}).
Z kolei wyniki dla 6 wątków są gorsze niż dla 4 i porównywalne z tymi dla 2.
Dodatkowo wyniki dla 4 wątków były bardzo zmienne.
Na wykresie zostały zamieszczone te najlepsze, wynoszące nieco poniżej 60s natomiast
zdarzały się również takie powyżej 100s.
Ciężko powiedzieć co jest przyczyną takiego zachowania programu.
Skalowanie między 1 i 2 wątkami jest prawie idealne natomiast kolejne wyniki
pozostawiają wiele do życzenia.
Nawet te dla 4 wątków ze względu na ich brak powtarzalności.

Pierwszą ważną rzeczą jest sposób działania programu.
Jest on identyczny z wersją C++.
Na początku procesy kopiują do siebie wejściową tablicę.
Dzięki temu (potwierdziły to również logi) jedyna komunikacja, która ma miejsce
to ta na początku programu (podczas kopiowania tablicy i trwa koło 1s dla 6 procesów)
oraz na końcu podczas sumowania wyników.
Poza tym procesy nie potrzebują się miedzy sobą komunikować dlatego można wykluczyć
narzut komunikacyjny jako przyczynę problemów z wydajnością.

Kolejny problem pojawia się przy zajmowaniu wątków systemowych.
Pomimo bezpośredniego wskazania, że jeden \texttt{locale} może wykorzystywać maksymalnie
jeden wątek w rzeczywistości wykorzystuje zawsze 2 razy więcej.
I tak uruchamiając program na 2 localach zajmujemy 4 rdzenie procesora, a
na 3 localach zajmujemy 6 rdzeni.
Pomimo tego, że na 4 localach również zajmujemy 6 rdzeni (ograniczenia sprzętowe)
to wynik potrafi być mimo wszystko uzyskany sporo szybciej.

W celu dokładniejszego zbadania problemu zmodyfikowałem przykład w taki sposób,
że każdy z wątków tworzy własną tablicę i wypełnia ją własnymi liczbami.
Nie dochodzi wtedy do kopiowania tablicy a co za tym idzie do komunikacji.
Przykład ten miał za zadanie sprawdzić czy poprawią się jedynie wyniki czasowe bo
z punktu widzenia użyteczności nie ma sensu - nie liczy żadnego wyznacznika.
Alternatywnie można było zaimplementować obliczanie liczby PI ale ze względów czasowych
łatwiej było zmodyfikować posiadany już kod.
Nietety wyniki okazały się podobne do tych poprzednich co potwierdza, że to nie komunikacja
jest problemem.

Jeszcze jedną, ciekawą obserwacją jest czas wykonania poszczególnych wątków.
Każdy wątek po wystartowaniu wypisuje odpowiednią informację na standardowe wyjście.
Niestety przy uruchamianiu np. na 4, 5 czy 6 rdzeniach informacje nie pojawiały się w tym
samym czasie.
Wyglądało to tak, że pojawiały się logi od np. 4 rdzeni, potem rdzenie te wykonywały swoją
pracę i zapisywały wyniki a dopiero na końcu startował ostatni rdzeń mimo, że zużycie
procesora w monitorze systemowym wskazywało na wykorzystanie wszystkich
5 rdzeni przez cały ten czas.

\begin{figure}
    \centering
    \begin{tikzpicture}
    \begin{axis}[
        xlabel=Liczba wątków,
        ylabel=Czas wykonania (s),
        width=\columnwidth,
        legend pos=north east]
        \addplot table[mark=*, x index=0, y index=1] {results/chapel-matrixdet.dat};
    \end{axis}
\end{tikzpicture}

    \caption{Czas wykonania algorytmu w zależności od liczby wątków}
    \label{fig:chapel-matrixdet-time}
\end{figure}

\begin{figure}
    \centering
    \begin{tikzpicture}
    \begin{axis}[
        xlabel=Liczba wątków,
        ylabel=Przyspieszenie,
        width=\columnwidth,
        legend pos=south east]
        \addplot table[mark=x, x index=0, y index=2] {results/chapel-matrixdet.dat};
        \addplot[domain=0:9, samples=2, color=gray, style=dashed]{x};
    \end{axis}
\end{tikzpicture}

    \caption{Przyspieszenie algorytmu w zależności od liczby wątków}
    \label{fig:chapel-matrixdet-speedup}
\end{figure}

\subsection{Podsumowanie}
Podsumowując warto skupić się na dwóch aspektach.
Pierwszym z nich jest prostota użytkowania.
Chapel wydaje się być bardziej podobny do Pythona.
UPC++ wymaga znajomości C++ co od razu sprawia, że jest trudniejszy do użycia.
Można powiedzieć, że pod tym względem Chapel wypada trochę lepiej.
Drugim aspektem jest wydajność i skalowalność.
I tutaj zdecydowanie wygrywa UPC++.
W moich badaniach Chapel okazał się nie tylko mało wydajnym językiem ale również
nieskalowalnym.
Być może jest to spowodowane złym wykorzystaniem Chapela.
Przeglądając dokumentację można wysnuć bardzo małe podejrzenie, że mechanizmu
\texttt{locale} powinno się używać dla różnych maszyn.
Być może to jest powodem braku skalowalności.
Nawet jeśli, to pozostaje jeszcze niska wydajność Chapela praktycznie
4-krotnie mniejsza niż UPC++.