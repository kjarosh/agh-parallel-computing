\section{Wyznacznik macierzy (Michał Śledź)}
\label{sec:matrix_det}

\subsection{Środowisko testowe}

\begin{description}
    \item [procesor]: i5-9600K 6 rdzeni, 3,70-4,60 GHz, 9MB Cache
    \item [pamięć]: 16 GB RAM, 3200 MHz, CL 16
\end{description}
Testy zostały wykonane na jednej maszynie lokalnej (opisanej powyżej) bez udziału
sieci.
Testy były uruchamiane wielokrotnie i dawały zbliżone do siebie wyniki.

\subsection{UPC++}

Na początku deklarujemy dwa distributed objecty, dzięki czemu
każdy z procesów będzie miał te same zmienne lecz z innymi
wartościami~\ref{lst:upcxx-dstobj-init}.

\begin{lstlisting}[
    caption=Inicjalizacja distributed objectów,
    label={lst:upcxx-dstobj-init}]
upcxx::dist_object<upcxx::global_ptr<double>>
    u_g(upcxx::new_array<double>(N*N));
upcxx::dist_object<upcxx::global_ptr<double>>
    sum(upcxx::new_<double>(0));
\end{lstlisting}

Następnie proces o id równym 0 inicjalizuje tablicę, dla której
mamy policzyć wyznacznik, a kolejne procesy kopiują
jej zawartość~\ref{lst:upcxx-state-sync}.

\begin{lstlisting}[
    caption=Ustalenie stanu procesów,
    label={lst:upcxx-state-sync}]
if (upcxx::rank_me() == 0) {
    init_matrix(&u_g);
}
upcxx::barrier();

upcxx::global_ptr<double> u =
    u_g.fetch(0).wait();
double **matrix = allocate_matrix(N);
for (int i = 0; i < N; i++) {
    for (int j = 0; j < N; j++) {
        int ptr = u + (i*N + j);
        matrix[i][j]=upcxx::rget(ptr).wait();
    }
}
\end{lstlisting}

Kolejnym krokiem jest obliczenie przez każdy z procesów
rozwinięcia Laplace'a, które oznaczone jest przez wywołanie metody run.
Każdy z procesów liczy wspomniane rozwinięcie tylko dla wybranych kolumn, zgodnie
ze wzorem:
\begin{equation}
    k = proc\_id + n*proc\_num
\end{equation}
Wywołujemy również funkcję \texttt{upcxx::barrier()} celem synchronizacji wszystkich
procesów w tym miejscu~\ref{lst:upcxx-laplace}.

\begin{lstlisting}[
    caption=Liczenie rozwinięcia Laplace'a,
    label={lst:upcxx-laplace}]
    double *local_sum = sum->local();
    *local_sum = run(matrix, N);
    upcxx::barrier();
\end{lstlisting}

Na końcu proces o id równym zero odczytuje zmienne sum pozostałych
procesów wyliczając w ten sposób ostateczną
wartość wyznacznika~\ref{lst:upcxx-sum-results}.

\begin{lstlisting}[
    caption=Sumowanie wyniku,
    label={lst:upcxx-sum-results}]
if (upcxx::rank_me() == 0) {
    int proc_n = upcxx::rank_n();
    for (int i = 1; i < proc_n; i++) {
        upcxx::global_ptr<double> u =
            sum.fetch(i).wait()
        *local_sum +=
            upcxx::rget(u).wait();
    }
    cout << "\ndet: " << *local_sum;
}
\end{lstlisting}

Wykres~\ref{fig:upcxx-matrixdet-time} przedstawia czas wykonania algorytmu w zależności
od liczby wykorzystanych wątków.
Rezultaty są spodziewane.
Dla 5 wątków nie uzyskujemy żadnego przyspieszenia w stosunku
do 4 co jest spowodowane tym, że rozmiar testowej macierzy nie jest podzielny przez 5.
W rezultacie w ostatniej iteracji trzy wątki będą bezczynne ale wciąż trzeba poczekać na
pozostałe dwa.
Stąd brak przyspieszenia.
Jest to dokładnie widoczne na wykresie~\ref{fig:upcxx-matrixdet-speedup}.

\begin{figure}[h]
    \centering
    \begin{tikzpicture}
    \begin{axis}[
        xlabel=Liczba wątków,
        ylabel=Czas wykonania (s),
        legend pos=north east]
        \addplot table[mark=*, x index=0, y index=1] {results/upcxx-matrixdet.dat};

        \legend{Czas wykonania}
    \end{axis}
\end{tikzpicture}

    \caption{Czas wykonania algorytmu w zależności od liczby wątków}
    \label{fig:upcxx-matrixdet-time}
\end{figure}

\begin{figure}[h]
    \centering
    \begin{tikzpicture}
    \begin{axis}[
        xlabel=Liczba wątków,
        ylabel=Przyspieszenie,
        width=\columnwidth,
        legend pos=south east]
        \addplot table[mark=x, x index=0, y index=2] {results/upcxx-matrixdet.dat};
        \addplot[domain=0:9, samples=2, color=gray, style=dashed]{x};
    \end{axis}
\end{tikzpicture}

    \caption{Przyspieszenie algorytmu w zależności od liczby wątków}
    \label{fig:upcxx-matrixdet-speedup}
\end{figure}
